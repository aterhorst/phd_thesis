\begin{landscape}
\footnotesize
\singlespacing
\begin{longtable}[c]{@{}ll p{6cm}cc@{}}
\caption[Codebook]{Semi-structured interviews codebook. \label{tab:codebook}} \\

\toprule
Code & Name & Description & Files & References \\ 
\endfirsthead

\multicolumn{5}{c}{Table \ref{tab:codebook} continued from previous page.}\\
\toprule
Code & Name & Description & Files & References \\ 
\midrule
\endhead

\bottomrule \\
\endfoot

\bottomrule
\endlastfoot

\midrule
1 & CC - INNATE NEEDS & Category code. & 15 & 23 \\
1.1 & PC -  SEEKING AUTONOMY & Individuals strive to become more autonomous. & 8 & 11 \\
1.2 & PC - DOING SATISFYING WORK & People are motivated by a desire to satisfy innate needs. Need for competence, autonomy and social connectedness. & 9 & 12 \\
2 & CC - INTERNALISED SOCIAL NORMS & Category code. & 13 & 51 \\
2.1 & PC - EXPRESSING A PARTICULAR WORLDVIEW & Individual needs dispositions and internalised social norms moderate a person's willingness to seek out or share tacit knowledge. How people see the world is a reflection of internalised social norms. & 9 & 28 \\
2.2 & PC - IDENTIFYING WITH A DISTINCT GROUP & Social identity may moderate tacit knowledge sharing. & 8 & 23 \\
3 & CC - MECHANISMS \& STRUCTURES & Category code. & 21 & 313 \\
3.1 & PC - BUILDING TRUST RELATIONS & Reciprocity and closure in tacit knowledge exchange networks indicate high levels of trust in open innovation partnerships. & 20 & 117 \\
3.2 & PC - DEFINING THE RULES OF ENGAGEMENT & Putting into place rules that govern interaction. & 2 & 4 \\
3.3 & PC - ENCOUNTERING POWER STRUCTURES & Dealing with more powerful organisations and people. 3A & 5 & 18 \\
3.4 & PC - PROTECTING KNOWLEDGE & Protecting intellectual property or commercial-in-confidence information. & 12 & 28 \\
3.5 & PC - SPANNING REAL AND PERCEIVED BOUNDARIES & Open innovation requires practitioners to connect across real and perceived boundaries to apply their know-how in novel ways. & 21 & 146 \\
4 & CC - ACTION & Category code & 21 & 531 \\
4.1 & PC - APPLYING KNOWLEDGE IN PRACTICE & Reducing cognitive distance between open innovation partners requires significant social interaction to support the application of knowledge in practice. & 19 & 130 \\
4.2 & PC - CONNECTING PEOPLE & Successful open innovation requires a combination of skilled brokerage and network closure. & 17 & 60 \\
4.3 & PC - EMPOWERING OTHERS & Who people choose to empower with their know-how depends on how much they trust the receiver to use their know-how in mutually beneficial ways. & 10 & 17 \\
4.4 & PC - FOSTERING COLLABORATION & Successful open innovation requires a combination of skilled brokerage and network closure. The combination of less brokerage and more closure indicates a higher level of collaboration. & 21 & 146 \\
4.5 & PC - PROMOTING A LEARNING CULTURE & Reducing cognitive distance between open innovation partners requires significant social interaction to support learning. & 20 & 111 \\
1.1.1 & DC - BECOMING MORE COMPETENT & People are driven to satisfy an innate need for competence or self-efficacy. This is empowering and helps them deal with the environment. & 7 & 9 \\
1.1.2 & DC - CONNECTING WITH OTHERS & Some people are motivated by an innate need for social relatedness. & 2 & 2 \\
1.2.1 & DC - PERFORMING MEANINGFUL WORK & People are motivated by a a desire to do meaningful work. This gives them a sense of purpose and is a great source of personal enjoyment and satisfaction. & 9 & 11 \\
2.1.1 & DC - DEMONSTRATING A SUPERIOR ATTITUDE & Belief that people without a background in science are incapable of understanding highly technical matters. & 4 & 6 \\
2.1.2 & DC - MAINTAINING A NARROW PERSPECTIVE & Some people are not that interested in what is happening outside their domain. & 6 & 17 \\
2.1.3 & DC - NAVIGATING CHANGE & Some people resist change. They are conservative and prefer things the way they are. & 2 & 3 \\
2.1.4 & DC - OVER-ANALYSING THINGS & Belief that educated people are prone to analysis paralysis. & 1 & 2 \\
2.2.1 & DC - BEING ABSOLUTELY COMMITTED & Committed to the success of a collaboration. & 8 & 23 \\
3.1.1 & DC - BREAKING TRUST & Bad behaviour breaks trust. & 5 & 9 \\
3.1.2 & DC - DEMONSTRATING CREDIBILITY & People are less likely to trust your input if you are not seen as a credible or authoritative source of knowledge. & 6 & 13 \\
3.1.3 & DC - DEMONSTRATING SWIFT TRUST & People tend to show presumptive trust when thrown together. Failure to meet expectations erodes this trust. & 7 & 13 \\
3.1.4 & DC - DEVELOPING TRUST & Building trust takes time. Acting in good faith helps build trust. & 0 & 0 \\
3.1.5 & DC - HAVING OPEN AND HONEST DISCUSSIONS & Having frank, open, and honest, discussions helps maintain and enhance trust. & 13 & 30 \\
3.1.6 & DC - INVESTING IN RELATIONSHIPS & Placing a premium on relations helps maintain trust. & 14 & 37 \\
3.1.7 & DC - SHOWING COMMITMENT & People are more likely to trust others who are highly committed. & 5 & 7 \\
3.1.8 & DC - TAKING RESPONSIBILITY & Taking responsibility for mistakes helps build trust and sustain healthy collaboration. & 4 & 4 \\
3.1.9 & DC - WITHHOLDING INFORMATION & People withhold information or knowledge from others they do not trust fully. & 4 & 4 \\
3.2.1 & DC - AVOIDING CONFLICTS & Avoid collaborating with others where this may result in a conflict of interest. & 1 & 2 \\
3.2.2 & DC - LACKING STRUCTURE & Absence of formal structure resulting in poor coordination and sub-optimal open innovation outcomes. & 1 & 1 \\
3.2.3 & DC - PLAYING BY THE RULES & Need to set clear rules that collaborators abide with. Failure to abide by the rules can compromise other collaborators. & 1 & 1 \\
3.3.1 & DC - DEALING WITH LARGE ORGANISATIONS & Large organisations hold more power over individuals or smaller entities. & 4 & 17 \\
3.3.2 & DC - EXERCISING CONTROL & Some people like being in charge or in control of things. & 1 & 1 \\
3.4.1 & DC - ENGAGING IN SUBVERSIVE BEHAVIOUR & Attempts to block access to knowledge can drive subversive behaviour. People bypass gatekeepers to gain access to important knowledge. & 3 & 3 \\
3.4.2 & DC - PROTECTING INTELLECTUAL PROPERTY & Protecting IP moderates knowledge sharing. & 10 & 19 \\
3.4.3 & DC - WITHHOLDING INFORMATION & People withhold commercially-sensitive information and knowledge from others they do not trust entirely. & 5 & 6 \\
3.5.1 & DC - COMMUNICATING POORLY & Poor communication can impede knowledge flows and lead to misunderstanding and erosion of trust, & 11 & 30 \\
3.5.10 & DC - WORKING WITH DIFFERENT PERSONALITIES & Personality conflicts can undermine collaboration and/or limit knowledge sharing. & 2 & 6 \\
3.5.2 & DC - DEALING WITH CHANGE & The prospect of having to change how things are done can be a boundary. & 10 & 18 \\
3.5.3 & DC - DEALING WITH FOREIGN CULTURES & A lack of awareness of cultural differences may undermine the ability to share knowledge freely. & 12 & 24 \\
3.5.4 & DC - DEALING WITH THE TYRANNY OF DISTANCE & Physical separation impedes knowledge sharing. Not only does this limit face-to-face interaction, time-differences can make communication difficult. & 14 & 22 \\
3.5.5 & DC - GAINING ACCESS TO KEY PEOPLE & Obtaining knowledge from busy people can be hard. Finding time to pin them down can be challenging and frustrate knowledge sharing. & 5 & 12 \\
3.5.6 & DC - LACKING RESOURCES & A lack of resources may undermine effective collaboration/knowledge sharing (e.g. finances, manpower, knowledge, and/or facilities). & 7 & 11 \\
3.5.7 & DC - ORGANISATIONAL BOUNDARIES & Intra- and inter-organisational boundaries can impede collaboration/knowledge sharing. & 10 & 14 \\
3.5.8 & DC - OVERCOMING DISCIPLINARY BOUNDARIES & Each discipline has its own way of doing things. This can undermine collaboration. & 6 & 7 \\
3.5.9 & DC - RECOGNISING AND MANAGING BOUNDARIES & Recognising boundaries can mitigate their negative impacts. & 2 & 2 \\
4.1.1 & DC - DEALING WITH COMPLEXITY & Applying knowledge to find novel solutions to complex or wicked problems. & 9 & 20 \\
4.1.10 & DC - BREAKING NEW GROUND & People get a greater sense of purpose doing ground-breaking work. & 11 & 32 \\
4.1.11 & DC - DEMONSTRATING A CAN-DO ATTITUDE & Not being daunted by potential obstacles. Willing to try different things to succeed. & 2 & 2 \\
4.1.12 & DC - PROFITING FROM INNOVATION & Innovation improves competitive advantage and business profitability. & 10 & 25 \\
4.1.13 & DC - RISING TO THE  CHALLENGE & People are motivated to tackle something novel and challenging. & 5 & 8 \\
4.1.2 & DC - IMPROVING PRACTICES & Applying knowledge in practice is often about continuous process improvement. & 9 & 36 \\
4.1.3 & DC - INTERACTING FACE-TO-FACE & Face-to-face interaction is crucial for applying know-how in practice. & 7 & 14 \\
4.1.4 & DC - PRACTISING INTEGRATED THINKING & To solve a complex problem requires integrated thinking, one that can handle multiple and often conflicting perspectives. & 3 & 7 \\
4.1.5 & DC - REFLECTING ON WHAT CAN WORK & Reflecting on what did work or what did not work and learning from this. & 10 & 26 \\
4.1.6 & DC - RELYING ON TACIT KNOWLEDGE & One has to tap into the know-how and experience of people to understand how things work in practice. & 3 & 6 \\
4.1.7 & DC - USING DATA TO GUIDE DECISIONS & Improving practices implies a need for evidence of improvement. & 5 & 9 \\
4.1.8 & DC - WORKING IN DIFFERENT CONTEXTS & Apply learning in different or new contexts. & 7 & 12 \\
4.1.9 & PC - APPLYING KNOWLEDGE INNOVATIVELY & Open innovation requires practitioners to connect across real and perceived boundaries to apply their know-how in novel ways. & 17 & 67 \\
4.2.1 & DC - ACTING AS A COMPANY REPRESENTATIVE & Representative broker role. Sharing external knowledge with internal stakeholders. & 6 & 15 \\
4.2.2 & DC - COORDINATING PEOPLE & Coordinator broker role. Circulating knowledge internally. & 1 & 1 \\
4.2.3 & DC - EXCLUDING OTHERS & Tertius guadens brokerage. Keeping people apart or excluding them from deliberations. & 3 & 6 \\
4.2.4 & DC - FORGING STRONG STAKEHOLDER RELATIONS & Tertius inungens brokerage. Strengthening ties. & 1 & 1 \\
4.2.5 & DC - GATEKEEPING & Gatekeeper broker role. Being the arbiter of internal knowledge dissemination. & 11 & 16 \\
4.2.6 & DC - PRACTISING LIAISON BROKERAGE & Liaison broker role. Passing knowledge from one external party to another external party. & 8 & 20 \\
4.2.7 & DC - RESOLVING CONFLICTS & Mediator broker role. Facilitating the flow of knowledge between two parties. & 1 & 1 \\
4.3.1 & DC - EMPOWERING CUSTOMERS & Strengthen relationships by helping customers benefit commercially from improved practices. & 2 & 4 \\
4.3.2 & DC - NEGOTIATING FROM A POSITION OF STRENGTH & Knowledge allows one to negotiate from a position of strength. & 2 & 3 \\
4.3.3 & DC - SHARING KNOWLEDGE WITH OTHERS & Sharing knowledge with others empowers them. & 7 & 10 \\
4.4.1 & DC - ACTING IN SELF-INTEREST & Actions designed to further self-interests. & 11 & 33 \\
4.4.10 & DC - WORKING IN ISOLATION & Working in silos is not conducive to effective collaboration. & 5 & 6 \\
4.4.11 & DC - WORKING TOWARDS A COMMON GOAL & Having a shared goal makes collaboration so much easier. & 15 & 32 \\
4.4.2 & DC - EMBRACING MULTIDISCIPLINARITY & Happy to take advice from others from different disciplines. & 6 & 10 \\
4.4.3 & DC - ENSURING EVERYONE BENEFITS & Collaborations are more likely to succeed when benefits are shared equally. & 8 & 9 \\
4.4.4 & DC - HAVING DIFFICULT CONVERSATIONS & Tension surfaces hidden or avoided problems that can be tackled jointly. & 3 & 6 \\
4.4.5 & DC - HAVING SUFFICIENT CRITICAL MASS & Collaborating allows one to apply more minds to a problem. & 2 & 2 \\
4.4.6 & DC - MANAGING EXPECTATIONS & People have different and often unrealistic expectations about what a collaboration can deliver. & 11 & 35 \\
4.4.7 & DC - OPERATING IN DIFFERENT GEOGRAPHIC MARKETS & Working in different markets makes it easier to share knowledge and trade secrets. & 1 & 1 \\
4.4.8 & DC - SUPPORTING OTHERS & Collaboration is often about supporting others, helping them achieve their objectives. & 6 & 8 \\
4.4.9 & DC - UNDERSTANDING CUSTOMERS & Proper understanding of customer practices makes it easier to empathise with their issues/problems. & 3 & 4 \\
4.5.1 & DC - CODIFYING KNOWLEDGE & Learning is captured when knowledge is codified. Codifying knowledge helps others learn. & 6 & 12 \\
4.5.2 & DC - FOSTERING A LEARNING CULTURE & Innovation requires a learning culture. People must be motivated to learn how to do things better. Learning also aids absorptive capacity. & 2 & 2 \\
4.5.3 & DC - GAINING A BROADER UNDERSTANDING & Solving problems requires a good understanding of contributory factors. & 12 & 26 \\
4.5.4 & DC - LEARNING BY DOING & Learning happens through applying knowledge in practice. Learn by doing is what drives self-efficacy. & 13 & 34 \\
4.5.5 & DC - TAKING SHORT CUTS & Figuring out how to take short-cuts is a form of learning. & 1 & 1 \\
4.5.6 & DC - TAPPING EXTERNAL EXPERTISE & Learn from others with appropriate expertise. & 16 & 36 \\ 
\end{longtable}
\end{landscape}