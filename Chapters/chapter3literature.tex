
% Literature review

- Open innovation is essentially about managing knowledge flows across organisational boundaries.
- This includes encouraging people to willingly share or seek out knowledge and making sure the organisation profits in some way from knowledge sharing
- Individual and organisational openness to learning and/or new knowledge is key here.
- Need to recognise knowledge is a form of power.
- Useful to understand how knowledge sharing behaviour is influenced by personal motivation, power-relations, and organisational factors.



\section{Motivation to share knowledge}

- What is motivation? Motivation is a theoretical construct used to explain behaviour.
- people are motivated to share knowledge either out of self-interest, altruism, or both. [Hsiu-Fen Lin, 2007]
- people motivated out of self-interest do so to empower themselves in two ways - satisfy some innate need or gain some personal advantage by being gatekeeper of knowledge.
- people motivated out of altruism want everybody, including themselves, to succeed. They succeed if everybody else succeeds.
- we can look at motivation in terms of self-determination theory (innate needs) and brokerage (behaviour, actions).
- Tacit knowledge requires more effort (i.e. higher levels of motivation) to communicate.  [osterloh and frey 2000]
- Past studies show that intrinsic motivation predicts tacit knowledge sharing e.g. 

\section{Self-determination theory}


- Many motivational theories. With innovation, interested on motivation for learning. 
- Innovation is intrinsically motivated. [Bhaduri & Kumar 2011]
- SDT is useful for assessing intrinsic and extrinsic motivation.
- SDT argues behaviour is driven by innate need for competence, autonomy and relatedness.
- SDT conceptualise motivation as a continuum comprising amotivation, extrinsic motivations, and intrinsic motivation.
- SDT suggest extrinsic motivations differ depending on locus of control.
- Material and introjected regulation locus is external. Integrated and identified regulation locus is internal.
- Intrinsic motivation is by definition internal.
- The more one internalises reasons for externally motivated actions, the more these actions become self-determined.
- Knowledge sharing behaviour can be explained in terms of TPB/SDT [Gagne 2009]
- TPB explains how attitudes, norms may influence externally regulated behaviour (extrinsic motivators).

- Proposition 1A: People with higher levels of autonomous motivation are more likely to share tacit knowledge
- Proposition 1B: People with higher levels of autonomous motivation are more likely to seek out tacit knowledge


\section{Brokerage}

- Define brokerage [ Marsden 1982 ]
- Brokerage roles [ Gould and Fernandez 1989]
- Strategic orientations - tertius gaudens [Burt] vs, tertius inungens [Obstfeld]
- Knowledge is power. Consider brokerage in terms of self-empowerment vs. empowering others [ soda, tortoriello, iorio 2017 ]
- Brokers in innovation networks

- Proposition 3: Power in open innovation collaborations can be characterised in terms of brokerage roles 

\section{Summary}

- Large part of managing knowledge flows in open innovation is about creating a climate conducive to knowledge sharing.
- Cannot coerce people to share knowledge - this is something that must be done through volition.
- Understanding what motivates people to share and/or broker knowledge is a good first step.
- TPB and SDT can help with this.
- Also important to consider knowledge as power.
- Next section explains how mixed method SNA can be used to assess knowledge sharing/brokerage behaviour and power relations. 



