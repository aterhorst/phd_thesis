% epistemological foundation

First, there is the argument that the positivist method itself distorts reality, by distancing those who study reality (the expert) from those who experience it through their own lived subjectivity. \citep{gaventa2007power}.

\citep{Research philosophy}

Philosophical debates tend to revolve around ontology and epistemology. Ontology is all about the nature of reality whereas epistemology is about the theory of knowledge that helps researchers make sense of reality. Reality can be treated as concrete, something that can be observed or measured. This is OK when dealing with the physical world. However, when considering human behaviour, reality is what people make it out to be. Human behaviour is guided by how individuals perceive their own reality. We are dealing with multiple realities. Researchers investigating physical phenomena are inclined to embrace a positivist epistemology. They consider themselves as detached observers, able to reduce the problem into measurable and unambiguous parts, and demonstrate causality through statistical and mathematical methods. Behavioural researchers are more likely to embrace a constructivist epistemology that can deal with complexity borne out of multiple realities \citep{easterby2015management}. This study embraces a pragmatist epistemology, one that uses whatever approach is appropriate for the problem at hand.


\begin{table}[]
	\centering
	\caption{Contrasting positivitism and social constructivism. Reproduced from \citep{easterby2015management}}
	\label{epistemology}
	\begin{tabular}{@{}lll@{}}
		\toprule
											& \textbf{Positivism}						 &\textbf{Constructivism}   \\ 
		\midrule
		\textbf{The observer}               & Must be independent                        & Is part of what is being observed                          \\
		\textbf{Human interests}            & Should be irrelevant                       & Are the main drivers of science                            \\
		\textbf{Explanations}               & Must demonstrate causality                 & Aim to increase the general understanding of the situation \\
		\textbf{Research progresses through}& Hypotheses and deductions                  & Gathering rich data from which ideas are induced           \\
		\textbf{Concepts}                   & Need to be defined so they can be measured & Should incorporate stakeholder concepts                    \\
		\textbf{Units of analysis}          & Should be reduced to the simplest terms    & May account for complexity of whole situations             \\
		\textbf{Generalisation through}     & Statistical probability                    & Theoretical abstraction                                    \\
		\textbf{Sampling requires}          & Large numbers selected randomly            & Small number of cases  chosen for specific reasons         \\ \bottomrule
	\end{tabular}
\end{table}



\begin{table}[]
	\centering
	\caption{My caption}
	\label{my-label}
	\begin{tabular}{@{}lllll@{}}
		\textbf{Ontologies}      & Realism                     & Internal realism              & Relativism                 & Nominalism                  \\
		\textbf{Epistemology}    & Strong positivism           & Positivism                    & Constructivism             & Strong constructivism       \\
		\textbf{Aims}            & Discovery                   & Exposure                      & Convergence                & Invention                   \\
		\textbf{Starting points} & Hypotheses                  & Propositions                  & Questions                  & Critiques                   \\
		\textbf{Designs}         & Experiments                 & Large surveys, multiple cases & Cases, surveys             & Engagement, reflexivity     \\
		\textbf{Data types}      & Numbers, facts              & Mainly numbers, some words    & Mainly words, some numbers & Discourse, experiences      \\
		\textbf{Interpretation}  & Verification, falsification & Correlation, regression       & Triangulation, comparison  & Sense-making, understanding \\
		\textbf{Outcomes}        & Confirmation of theories    & Theory testing and generation & Theory generation          & New insights and actions   
	\end{tabular}
\end{table}

The author was motivated to study open innovation because of an innate desire to become a more effective research manager. He wanted to understand how one can deliver radical innovations through collaborative research and development. The author admits his psychological attachment to the subject matter does not allow him to be a completely detached observer. This should not be perceived negatively as in every situation observation is inflected with the values and beliefs of the observer \citep{monahan2010benefits}.

All knowledge is contingent on the interests of those creating it, the tools and procedures they use to measure the phenomena under investigation, and the analytic frameworks they use to interpret their results


This was especially true regarding his analysis and interpretation of qualitative data. He found it hard not to picture himself in situations described by others and interpret these in terms of his own experiences.  

Coming from a natural science background, the author knew little about social science at the beginning of this study. He had many years industry experience and perceived himself to be a generalist who enjoyed seeking out novel solutions to a wide range of complex problems. The author has never been beholden to any particular approach to problem-solving and uses whatever technique he deems appropriate at the time. He considers himself a pragmatist, somebody who believes research is about finding solutions to real-world problems. A successful outcome is when research outputs translate directly into changes in practice.

Given the potential for observer bias and his opportunistic approach to problem-solving, the author did not think realised a post-positivist approach would not work 

Not surprisingly, the author adopted the same problem-solving approach in this study. His philosophical world-view is one of pragmatism, which is concerned with what works in practice and solutions to problems \citep{tashakkori2010sage,creswell2013research}. What is particularly appealing about pragmatism is its practicality, not so much its broader philosophical basis \citep{morgan2014pragmatism}. Pragmatism can deal with more than one reality, is not confined to a specific view-point, and is able to handle multiple methods \citep{johnson2004mixed}. Table \ref{philosophy} highlights key differences between different world-views.

\begin{table}
	\captionsetup{font=scriptsize}
	\centering
	\caption{Philosophical world-views and their practical implications. Reproduced from \citep{creswell2011designing}.}
	\label{philosophy}
	\resizebox{\textwidth}{!}{%
		\begin{tabular}{@{}lllll@{}}
			\toprule
			& \multicolumn{1}{c}{Postpositivism}	& \multicolumn{1}{c}{Constructivism} & \multicolumn{1}{c}{Participatory} & \multicolumn{1}{c}{Pragmatism} \\ \midrule
			Ontology	& Single reality 			& Multiple realities & Political reality & Single and multiple realities \\
			Epistemology& Distance and impartiality & Closeness & Collaborative & Practical \\
			Axiology	& Unbiased 					& Biased 	& Negotiated 	& Multiple stances \\
			Methodology	& Deductive 				& Inductive & Participatory 	& Combinatorial \\
			Rhetoric	& Formal 					& Informal 	& Advocacy 		& Formal or informal \\ \bottomrule
	\end{tabular}}
\end{table} 

Many researchers are conditioned to think only in terms of a post-positivist approach and tend to be critical or dismissive of alternative research paradigms, possibly reflecting an unconscious negative bias towards alternative world-views \citep{staller2013epistemological}.  

The author did not consider himself a completely detached observer and was cognisant of potential observer bias. This was especially true regarding his analysis and interpretation of qualitative data. He found it hard not to picture himself in situations described by others and interpret these in terms of his own experiences. 


He understood his limited understanding of social science and the evolving nature of open innovation required a flexible approach to research. Moreover, the author realised early on that it would be difficult to find organisations willing to participate in this study. He did not have the luxury of choice and his approach would have to be guided by emerging ethical and commercial sensitivities.

The author prefers to see the world as constructed, interpreted, and experienced by people in their interactions with each other and with wider social systems. He embraces an inductive approach to research, the aim being to understand a particular phenomenon within a given context. 

This world-view lends itself to an emergent research design

Embracing such a world-view is not without risk. Curiosity led the author up many blind alleys and frustrated his supervisors, who wanted him to nail down his research questions early on. The author had to develop a working knowledge of social science. With each new insight, his perception of the problem altered. Open innovation is a moving target This was exacerbated by the open innovation being a continually evolving concept.

, which had a knock-on effect on his research design.


Many researchers are conditioned to think only in terms of a positivist approach. They tend to be critical or dismissive of research not rooted in the positivist paradigm, possibly reflecting an unconscious negative bias towards alternative world-views \citep{staller2013epistemological}.  

