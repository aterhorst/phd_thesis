


% brokerage as an organising principle. Explains case 1 and 2, not all of case 3.
% brokerage explains differences between cases.
% brokerage roles provide insight into absorptive capacity processes.
% case 3 brokerage explains ideation but not knowledge provision. Indicative of dysfunction.
% brokerage enables creative process (lingo2010nexus)
% formalisation of open innovation practices (rangus 2016)

Recruiting appropriate open innovation partners is a challenge. Ideally, selection criteria should be based on the technical competence and the complementarity of partner resources. Cultural alignment, adaptability, long-term strategic orientation, organisational capabilities, economic performance, financial stability, mutual awareness of each other's capabilities and technological transparency also need to be considered \citep{aloini2016structured}.\medskip

People who are both strong knowledge producers and great collaborators enhance their firm’s innovative output \citep{grigoriou2014structural}. Relational stars possess strong combination of both human and social capital, that is, a strong individual-level productivity performance combined with a highly consequential network position \citep{grigoriou2014structural}.\medskip

\subsection{Overhead}

Though connectivity and teamwork are considered key to organisational success, this comes at a cost. The communication overhead of collaboration can become a significant burden \citep{cross2013has}. Recent studies show that the time spent by people engaged in collaborative activities has grown by 50\% over the past two decades. Employees spend up to 80\% of their time in meetings or answering colleagues’ requests, leaving little time for doing critical work on their own \citep{cross2013has,cross2015investing,cross2016collaborative}. 

\citet{cross2016collaborative} distinguish three types of collaborative resources: "informational resources" are knowledge and skills that can be recorded or passed on, "social resources" involving one's awareness, access and position in the network, which can be used facilitate collaboration, and "personal resources" that include one's own skills, time, and energy. Whereas informational and social resources are more easily shared, there is a finite amount of time a person can contribute to any collaborative activity \citep{cross2016collaborative}. Burn-out, disengagement can lead to under-performing networks \citep{cross2013has}.

This is evident in Cases 1 and 3 ...

Leaders can solve this problem in two ways: by streamlining and redistributing responsibilities for collaboration and by rewarding effective contributions \citep{cross2016collaborative}. 

Boundary-spanning networks are semi-formal in the sense these still need to be properly managed \citep{cross2015investing}. These networks need to be carefully cultivated. This may include reconfiguring such networks over time. Once networks are established, mechanisms are needed to track and measure both network performance and collaboration outcomes. It is important to identify where connectivity can produce value, rather than shooting blindly with interventions that simply layer on collaborative demands \citep{cross2015investing}.

Although open innovation has been widely adopted, the concept is still evolving. Most studies have emphasised the upsides of open innovation and paid little attention to the downsides \citep{hossain2013open}. Past studies have tended to focus on open innovation involving large multinational firms. 