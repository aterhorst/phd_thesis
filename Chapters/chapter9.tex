\section{Thesis summary}

This section presents an overview of this thesis. It highlights the importance of tacit knowledge for open innovation and explains how a network perspective of open innovation can offer valuable insight into tacit knowledge sharing behaviour. The overview then considers how the interaction between agency and structure may affect such behaviour. It then explains how critical realism adds rigour to mixed-method social network analysis before presenting the main thesis findings. \medskip   

A growing number of companies are turning to open innovation to stay competitive \citep{stanko2017under}. Open innovation is formally defined as \textquote{a distributed innovation process involving the purposive management of knowledge flows across organisational boundaries using pecuniary and non-pecuniary mechanisms in line with the firm's business model} \citep{chesbrough2014explicating}. The innovation process relies heavily on accessing, absorbing, and harnessing knowledge across organisational boundaries with open innovation \citep{chesbrough2012open}. Tacit knowledge refers to the knowledge, skills, and abilities an individual gains through experience that is difficult to put into words or otherwise communicate \citep{kreutz2014catalyzing}. It is strongly implicated in innovation as it guides the learning and thought processes that produce novel ideas \citep{leonard1998role,lam2000tacit}. Despite the importance of tacit knowledge for innovation, it has received scant attention in the open innovation literature. \medskip

Open innovation implies extensive inter-organisational relationships to gain access to new external knowledge and exploit novel ideas \citep{chiaroni2011open}. Proper understanding of open innovation processes requires a network rather than a firm-level perspective \citep{fichter2009innovation,martinez2014analysis,yun2016network}. This thesis treats an open innovation partnership as a temporary knowledge network deliberately set up to achieve a specific innovation outcome \citep{turner2003natureoftheproject,terhorst2018tacit}. It aims to improve our understanding of social mechanisms that enable or inhibit tacit knowledge sharing in open innovation by addressing four research questions: (1) What does the structure of tacit knowledge networks reveal about knowledge enacted in practice? (2) Does brokerage differ according to the type of knowledge exchanged? (3) To what extent does self-determination drive tacit knowledge sharing in open innovation? (4) What does the micro-structure of tacit knowledge networks reveal about trust and power relations in open innovation partnerships? Open innovation entails helping others develop the ability to apply knowing and knowledge in new and different contexts. Answers to these research questions should help people devise more effective strategies for bringing know-how and know-what together. \medskip

Sharing know-how is primarily an act of volition or free will \citep{polanyi1966tacit,hubrich2015embodiment,zhang2016critical}. Hence, this thesis assumes human agency lies at the heart of tacit knowledge exchange \citep{phelps2012knowledge}. Finding the right balance between structure and agency is a challenge in open innovation. Too much structure may inhibit individual willingness to share tacit knowledge or contribute ideas. Conversely, too little structure can make goals less clear, leading to unsatisfactory open innovation outcomes \citep{davis2010agency}. The academic literature suggests that human agents are primarily motivated to acquire tacit knowledge to satisfy an innate need for competence and gain sufficient power to maintain or enhance their autonomy, influence agendas, and enact change \citep{white1959motivation,bandura1977self,deci1989self}. External factors that affect tacit knowledge exchanges include subjective and behavioural norms, real or perceived boundaries, rules of engagement, and trust and power relations \citep{parsons1937structure,loyal2001agency}. The literature also suggests brokers have a crucial role in helping individuals overcome boundaries, build trust, and manage power relations \citep{obstfeld2014brokerage,quintane2016brokers}. \medskip

This thesis used a combination of statistical social network analysis and semi-structured interviews to investigate the emergence of tacit knowledge network structures in three open innovation case studies. It used a critical realist approach to interpret quantitative and qualitative results \citep{bhaskar2013realist}. Most mixed-method studies embrace a pragmatic approach, where the goal is to obtain beneficial results, even though this involves alternating between paradigms \citep{creswell2011designing}. The stratified ontology of critical realism allows for the legitimate combination of qualitative and quantitative methods \citep{giddings2006mixed,mcevoy2006critical}. A critical realist approach is particularly well-suited to a multiple case study approach where different contexts are in play \citep{welch2011theorising}. \medskip

The statistical social network analysis addressed observed or empirical reality. In contrast, the semi-structured interviews addressed unobserved or actual reality (the reality that shapes the emergence and functioning of knowledge provider networks). The statistical network analysis used exponential random graph models (ERGMs) to infer social processes in each case study's tacit and explicit knowledge provider networks \citep{robins2007introduction}. At the same time, the qualitative analysis of semi-structured interviews allowed a more profound exploration of the mechanisms and structures affecting tacit knowledge sharing. Applying the critical realist logic of retroduction and retrodiction to integrate the quantitative and qualitative results provided a more complete, expansive and diverse picture of the social mechanisms and structures affecting tacit knowledge sharing \citep{zachariadis2013methodological,mcavoy2018critical}. \medskip

One set of ERGMs examined the role of motivation, trust, and power in tacit and explicit knowledge provider networks. Another set of ERGMs looked at broker roles in both the tacit and explicit knowledge provider networks. Results from the first set of ERGMs underscored the importance of tacit knowledge in open innovation. Path closure in a social network reflects a human tendency to form groups. Two of the partnerships had a significant path closure in their tacit knowledge networks. No such effect is evident in the explicit knowledge networks. In other words, tacit knowledge features strongly in group work. A significant receiver effect for autonomous motivation in each partnership's tacit knowledge network indicates that autonomous motivation is a reliable predictor of learning intent or knowledge-seeking behaviour. The second model runs indicate that path closure and broker roles can account for virtually all observed network configurations. Analysis of semi-structured interviews indicates that tacit knowledge is often under-valued, negatively impacting problem-solving and innovation. Moreover, the qualitative analysis indicates that brokers can profoundly affect the application of knowledge in practice. Tension was evident in those partnerships dominated by one individual. Trust features strongly in most decisions to share tacit knowledge. Perceived acts of self-interest erode trust and contribute to partner tension. Generally, people are less likely to disclose their hard-earned tacit knowledge in low-trust situations. \medskip

This thesis discovered that tacit knowledge resides in various internal and external communities of practice and that brokers are crucial for connecting these different communities. It confirmed that individuals are primarily motivated to seek out tacit knowledge to improve their level of self-determination. However, this thesis also found that close-minded individuals or individuals who consider themselves superior are less likely to connect with external communities of practice. The ability to tap external communities of practice is affected by several mechanisms, including actual or perceived boundaries and power and trust relations. This thesis also discovered that individuals are more likely to share their tacit knowledge if this delivers some personal benefit. It found that successful open innovation requires partners to invest in relationships to facilitate open and honest discussions vital for practical problem solving and innovation.

\section{Research contribution}

\subsection{Theoretical contribution}

This thesis demonstrated the application of agency theory in an open innovation context. \citeauthor{loyal2001agency}'s \citeyearpar{loyal2001agency} agency model was used to explore how the causal powers of individuals and social structures interact to affect tacit knowledge sharing. The model was framed in terms of self determination theory \citep{deci1985conceptualizations}, the theory of planned behaviour \citep{ajzen1985intentions}, social exchange theory \citep{homans1961social,blau1964exchange}, rational choice theory \citep{parsons1937structure,coleman1990foundations}, and structuration theory \citep{giddens1984constitution}. Results from the ERGM analysis show that self-determination theory can explain tacit knowledge seeking behaviour in all three cases. The theory of planned behaviour implies that knowledge sharing and seeking behaviour is dependent on individual attitudes and subjective norms \citep{gagne2009model}. Analysis of semi-structured interviews reveals that attitudes and subjective norms impact tacit knowledge sharing and seeking behaviour in positive and negative ways. Participants who were deeply committed to partnership goals were more likely to exchange know-how. Those with a superior attitude or who were close-minded were less likely to exchange know-how. According to social exchange theory, a person does another a favour with a general expectation of some future non-binding return \citep{homans1961social}. This thesis did not find strong evidence of reciprocity in any of the tacit knowledge provider networks. Perhaps this reflects a difference in relative absorptive capacity, where individuals seeking tacit knowledge are less likely to reciprocate \citep{lane1998relative}. \medskip

One important discovery was the fact it is possible to explain the micro-structure of tacit and explicit knowledge provider networks by simply modelling broker roles. ERGM parameter estimates are usually seen to represent a kind of average across the entire graph. The ERGM analysis of broker roles suggests that such generalisation misses important nuance. Broker roles represent distinct triadic structures distinguished by group membership and are likely to be subsuming other more general triadic structures \citep{gould1989structures}. Modelling of broker roles may be providing a more nuanced and precise explanation of the structure of inter-organisational knowledge networks. 

\subsection{Methodological contribution}

This thesis breaks new ground by applying a critical realist approach to an ERGM study. Past ERGM studies have been positivist or pragmatic. Exponential random graph modelling can distinguish between ties formed due to actor attributes (e.g. their level of autonomous motivation) or whether an actor's centrality is the result of being embedded within other purely structural network structures (e.g. broker position). However, we also need qualitative data to gain a complete understanding of contextual factors affecting the formation of social structures \citep{welch2011theorising,bellotti2014qualitative}. Past ERGM studies have used qualitative data in a pragmatic way, with little attention given to ontology or epistemology \citep[e.g.][]{lomi2013focused,lusher2014cooperative}. Applying a pragmatic approach can be very challenging, especially when attempting to make sense of discordant data collected under different epistemological assumptions \citep{johnson2004mixed, giddings2006mixed, shannon2016making}. Multiple case studies make the challenge harder \citep{welch2011theorising}. The author wanted to be rigorous and, consequently, embraced a post-positivist approach based on critical realism. \medskip

This thesis embraced the stratified ontology of critical realism to examine how agency and structure interact in open innovation \citep{bhaskar2013realist}. Our knowledge provider networks represent observed or empirical reality. The ERGM analysis revealed network-specific causal mechanisms in our networks (e.g. path closure and autonomous motivation receiver effects). These network-specific mechanisms are likely to interact with other causal mechanisms operating in unobserved or actual reality. The semi-structured interviews helped uncover these other causal mechanisms. Applying the critical realist logic of retroduction and retrodiction to integrate the quantitative and qualitative results provided a more complete and nuanced picture of the social mechanisms and structures affecting tacit knowledge sharing in our three cases. This thesis successfully adapted the critical realist approach outlined by \citet{mcavoy2018critical} and paves the way for future mixed-method social network analysis.

\subsection{Practice contribution}

This study is concerned with tacit knowledge-sharing behaviours in open innovation projects. A key question is to what extent are these findings of relevance to open innovation management. Our revised propositions provide some management guidance in this respect. The central management takeaways are as follows: 

\noindent
Tacit knowledge is strongly implicated in innovation. It guides the thinking that gives rise to novel ideas \citep{leonard1998role,lam2000tacit,senker2008contribution}. Most of the participants interviewed acknowledged that tacit knowledge was essential. However, it appears tacit knowledge was under-valued in Case 1 and Case 3. Devising ways to connect know-how and know-what did not feature in either case. People leading open innovation initiatives need to recognise the importance of tacit knowledge for innovation and ensure appropriate measures are in place to maximise tacit knowledge exchanges.\medskip
\noindent
Managers who allow participants greater autonomy can expect more positive tacit knowledge-sharing behaviours in open innovation partnerships. \citep{gagne2005self,gagne2009model,longo2017struggling}. Facilitating the sharing of know-how enables others to learn the practice that entails the know-how \citep{cook1999bridging}. Limiting autonomy through overly top-down or prescriptive management may result in sub-optimal open innovation outcomes.\medskip
\noindent
Managers should identify tangible and intangible boundaries that may potentially affect knowledge sharing upfront. They can then put into place processes for overcoming these (e.g. cultural awareness training, employing language interpreters, planning more face-to-face meetings, arranging social events so that participants can get to know each other better, and so forth). Managers must guard against participants who put their interests ahead of the partnership or are not good team players. At the same time, managers must accept that openness is likely to vary among open innovation partners \citep{dahlander2010open}. Managers must also be sensitive to power imbalances in open innovation partnerships. Poor teamwork, lack of openness, and power imbalances can negatively impact open innovation by distorting or inhibiting tacit knowledge-sharing relations.\medskip
\noindent
Managers need to understand as well as explain how participants or partners will benefit from exchanging tacit knowledge. Less open partners may engage in some minor gatekeeping, but they may practise tertius gaudens brokerage in more extreme situations that can break trust \citep{obstfeld2014brokerage}. Focusing on relationship building will help build trust and contribute to greater openness. Managers must also ensure that everybody commits to a shared goal. Having partners commit to a shared goal contributes to a greater sense of camaraderie and helps build trust. One can expect better open innovation outcomes when partners are equally committed. There is merit in encouraging tertius inungens brokerage from the outset. Apart from facilitating and strengthening new connections, tertius inungens brokerage can help balance power relations and mediate a shared goal \citep{chesbrough2012open}.\medskip
\noindent
\citeauthor{loyal2001agency}'s \citeyearpar{loyal2001agency} agency model is a useful conceptual framework for assessing tacit knowledge sharing behaviour in open innovation partnerships. Mangers can use this model as a risk management tool to understand how innate needs, beliefs and attitudes, mechanisms and structures, and individual actions may affect tacit and explicit knowledge sharing.\medskip

Ultimately, the job of the open innovation manager is to find the right balance between structure and human agency. Too much structure may inhibit individual willingness to share tacit knowledge or contribute ideas. Conversely, too little structure can make goals less clear, leading to unsatisfactory open innovation outcomes. The key to finding the right balance is to get open innovation partners to commit to a common objective and allow human agency to thrive. Agency is what drives tacit knowledge sharing that is so critical for good innovation outcomes.

 \section{Study limitations}
 
 Each of the three cases investigated in this thesis was tackling a different innovation challenge. The type of open innovation being performed in each case was also different. We also need to consider that Cases 1 and 3 were at an early stage, while Case 2 was winding up at the time of data collection. Given these stark differences and the cross-sectional nature of this study, one must be careful comparing cases. For example, we expect brokerage to feature more strongly in the early stages of open innovation, and we see this in our ERGM results. However, we cannot infer too much about the transient nature of brokerage from our cross-sectional data. Understanding how brokerage plays out over time would require the analysis of longitudinal data. \medskip

Tacit knowledge is hard to quantify, and the criteria used to determine how much of the knowledge was tacit in terms of codifiability, observability, and complexity are basic. Survey respondents are likely to have interpreted these criteria differently. The predominantly explicit and predominantly tacit knowledge provider networks are likely to have some internal statistical variation. \medskip

The semi-structured interview questions were formulated while the quantitative data was still being collected. In hindsight, it would have been preferable to develop the semi-structured interview questions after completing the exponential random graph modelling. That way, the questions would have been more focused on seeking explanations to observed patterns of social interaction. Moreover, the semi-structured interviews covered only a fraction of the participants who responded to the online survey(33\%, 32\% and 18\% of survey respondents for each case, respectively). The interviewees did provide valuable information about critical factors affecting tacit knowledge exchanges. However, it was impossible to interview participants from all partner organisations, resulting in an incomplete picture. Furthermore, both the online survey and semi-structured interview questions were in English. Two of the three cases involved participants who spoke English as a second language. The extent to which this affected their understanding of, and response to, questions is not known. \medskip

Most academics accept that qualitative analysis is both interpretive and subjective \citep{aspers2019qualitative}. Each researcher brings their unique perspective when identifying and developing concepts and insights through close examination and reflection of their qualitative data. While the insights gained from the qualitative analysis helped the author make sense of his network data, he recognises that others may draw slightly different conclusions than what he did. The qualitative analysis would have yielded more insights had it been possible to have other researchers involved.

\section{Future research}

Exploratory studies such as this one open up many avenues for further research. Additional case studies would improve our understanding of agency and structure in open innovation. Follow-up case studies should target open innovation projects that are not only at a similar evolutionary stage but also where the type of open innovation is similar (vis-\`a-vis inbound, outbound, or coupled open innovation). Future studies might also consider a longitudinal network study of knowledge-sharing behaviour across multiple open innovation projects. A longitudinal study will shed light on the transient nature of brokerage and the emergence of informal structures deemed necessary for innovation. Further modelling of broker roles in other open innovation knowledge networks would confirm if broker roles alone can provide a more nuanced and precise explanation of network structure. \medskip

This thesis argues that connecting diverse communities of practice is essential for open innovation. However, we did not see evidence of reciprocal sharing of tacit knowledge between partner organisations. How different communities of practice work across organisational boundaries and learn from each other is worthy of more attention. There is also a need to investigate more robust measures of tacit knowledge. As mentioned above, measures such as codifiability, observability, and complexity are basic. Future studies should consider how much tacit knowledge is embodied in individuals and inculcated in group practice and culture. A rough indicator of tacit knowledge embodied in individuals is relevant work experience, but perhaps measures of \textquote{practical intelligence} are a better indicator \citep{hedlund2002tacit}. Such measures may reveal more about how individuals are able to share their know-how effectively to enable others to learn the practice that entails the know-how \citep{van1986central, cook1999bridging}. \medskip 

Future studies should try and build upon the novel critical realist approach used in this thesis. This approach allowed the author to understand how, and more importantly, why tacit knowledge sharing ties form the way they do in each network. This thesis adopted the critical realist process for applied business research devised by \citet{mcavoy2018critical}. There is a need to further refine the critical realist process used in this thesis and develop it into a more formalised approach for mixed-method social network analysis that others can use. 

% Thank you and goodbye. Now fuck off.