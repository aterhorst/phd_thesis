\section{Thesis summary}

A growing number of companies are turning to open innovation to stay competitive. With open innovation, the innovation process relies heavily on accessing, absorbing, and harnessing knowledge across organisational boundaries. Tacit knowledge refers to the knowledge, skills, and abilities an individual gains through experience that is often difficult to put into words or otherwise communicate. It is strongly implicated in innovation as it guides the learning and thought processes that produce novel ideas. Despite the importance of tacit knowledge for innovation, it has received scant attention in the open innovation literature. This thesis explores the tacit dimension of open innovation. \medskip

Open innovation implies extensive inter-organisational relationships to gain access to new external knowledge and exploit novel ideas. Proper understanding of open innovation processes requires a network rather than a firm-level perspective. This thesis treats an open innovation partnership as a temporary knowledge network deliberately set up to achieve a specific innovation outcome. It aims to improve our understanding of social mechanisms that enable or inhibit tacit knowledge sharing in open innovation by addressing four research questions: What does the structure of tacit knowledge networks reveal about knowledge enacted in practice? Does brokerage differ according to the type of knowledge exchanged? To what extent does self-determination drive tacit knowledge sharing in open innovation? What does the micro-structure of tacit knowledge networks reveal about trust and power relations in open innovation partnerships? \medskip

As tacit knowledge sharing is primarily an act of volition or free will, this thesis assumes that human agency drives the emergence of tacit knowledge networks. Finding the right balance between structure and agency is a challenge in open innovation. Too much structure may inhibit individual willingness to share tacit knowledge or contribute ideas. Conversely, too little structure can make goals less clear, leading to unsatisfactory open innovation outcomes. The academic literature suggests that agents are primarily motivated to acquire tacit knowledge to satisfy an innate need for competence and gain sufficient power to maintain or enhance their autonomy, influence agendas, and enact change. External factors such as subjective and behavioural norms, real or perceived boundaries, rules of engagement, and trust and power relations are likely to influence tacit knowledge exchanges. The literature also indicates that brokers have a crucial role in helping individuals overcome boundaries, build trust, and manage power relations. \medskip

This thesis used a combination of exponential random graph modelling (ERGMs) and semi-structured interviews to investigate how agency drives the emergence of tacit knowledge network structures in three open innovation case studies. Mixing quantitative and qualitative methods is challenging because of the complex ontological and epistemological issues involved. This thesis employed a critical realist methodology to interpret quantitative and qualitative results. The stratified ontology of critical realism allows for the legitimate combination of qualitative and quantitative methods. It is particularly well-suited to a multiple case study approach where different contexts are in play. ERGMs were used to infer social processes in tacit knowledge provider networks. Qualitative analysis of semi-structured interviews allowed a more profound exploration of the mechanisms and structures affecting tacit knowledge sharing. Applying the logic of retroduction and retrodiction to integrate the quantitative and qualitative results provided a more complete, expansive and diverse picture of the social mechanisms and structures affecting tacit knowledge sharing. \medskip

One set of ERGMs examined the role of motivation, trust, and power in tacit and explicit knowledge provider networks. Another set of ERGMs looked at broker roles in both the tacit and explicit knowledge provider networks. Results from the first set of ERGMs underscored the importance of tacit knowledge in open innovation. Path closure in a social network reflects a human tendency to form groups. Two of the partnerships had a significant path closure in their tacit knowledge networks. No such effect is evident in the explicit knowledge networks. In other words, tacit knowledge features strongly in group work. A significant receiver effect for autonomous motivation in each partnership's tacit knowledge network indicates that autonomous motivation is a reliable predictor of learning intent or knowledge-seeking behaviour. The second set of model runs indicates that path closure and broker roles can account for virtually all observed network configurations. Analysis of semi-structured interviews provided a deeper understanding of how agency and structure affect tacit knowledge sharing in open innovation. Results indicate that tacit knowledge is often under-valued, which can impact negatively problem-solving and innovation. Brokers can have a profound effect on the application of knowledge in practice. Tension was evident in partnerships dominated by one individual, attributed to the fact that some partners are wary of a highly central actor dominating the discourse. Trust features strongly in most decisions to share tacit knowledge. Perceived acts of self-interest erode trust and contribute to partner tension. People are less likely to disclose their hard-earned tacit knowledge in low-trust situations. \medskip

This thesis discovered that tacit knowledge resides in various internal and external communities of practice and that brokers are crucial for connecting different communities of practice. It confirmed that individuals are primarily motivated to seek out tacit knowledge to improve their level of self-determination. However, this thesis also found that individuals who are close-minded or consider themselves superior are less likely to connect with external communities of practice. The ability to tap external communities of practice is affected by several mechanisms that may include real or perceived boundaries and power and trust relations. This thesis also discovered that individuals are more likely to share their tacit knowledge if this delivers some benefit. It found that successful open innovation requires partners to invest in relationships to facilitate open and honest discussions so vital for effective problem solving and innovation.

\section{Research contribution}

\subsection{Theoretical contribution}

This thesis explored the tacit dimension of open innovation. It aimed to shed light on causal mechanisms driving the emergence of tacit knowledge network structures in open innovation partnerships. Because of its exploratory nature, the theoretical contribution of this thesis is limited. The initial set of propositions are informed by self determination theory \citep{deci1985conceptualizations}, the theory of planned behaviour \citep{ajzen1985intentions}, social exchange theory \citep{homans1961social,blau1964exchange}, rational choice theory \citep{parsons1937structure,coleman1990foundations}, and structuration theory \citep{giddens1984constitution}. These sociological theories attempt to explain how the social world operates. They are meta-theories insofar as they explain behaviours, interactions, and patterns of social organisation among humans at a broad scale \citep{turner2012nature}. This thesis did not set out to prove or disprove any of the theories. \medskip

However, this thesis does show that self-determination theory can explain some of the observed knowledge sharing behaviour, namely that autonomously motivated actors are more likely to seek out tacit knowledge. The theory of planned behaviour implies that an individual's motivation to seek out or share tacit knowledge is not only dependent on their perceived self-efficacy, it is also dependent on their attitudes and subjective norms \citep{gagne2009model}. This study shows that attitudes and subjective norms do indeed affect tacit knowledge sharing and seeking behaviour. According to social exchange theory, a person does another a favour with a general expectation of some future non-binding return \citep{homans1961social}. We do not see strong evidence of reciprocity in any of the tacit knowledge provider networks. Perhaps this reflects a teacher-student relationship of sorts. Individuals seeking tacit knowledge are less likely to be able to reciprocate. \medskip

This thesis demonstrated how agency theory, specifically rational choice and structuration theory, could be applied in an open innovation context. \citeauthor{loyal2001agency}'s \citeyearpar{loyal2001agency} agency model was particularly helpful in this respect. The model allowed one to explore how the distinctive causal powers of individuals and social structures interact and affect tacit knowledge sharing. One can use this model as a risk management tool in open innovation (to understand how innate needs, beliefs and attitudes, mechanisms and structures, and individual actions may affect tacit and explicit knowledge sharing). 

\subsection{Methodological contribution}

Whereas social network analysis is a powerful tool for describing patterns of social interaction, it cannot account for many of the exogenous factors that shape social ties. We need a combination of quantitative and qualitative methods to obtain a complete understanding of the mechanisms and structures affecting the formation of social structures. Any study that involves mixed methods must reconcile ontological and epistemological differences to ensure findings are both credible and valid. The critical realist approach used in this study addresses ontological and epistemological differences in a rigorous manner. The author is unaware of any other case-based network study that has used this approach. This study demonstrates how a critical realist approach adds rigour to mixed-method social network analysis.

\subsection{Practice contribution}

This study is concerned with tacit knowledge-sharing behaviours in open innovation projects. A key question is to what extent are these findings of broader relevance to open innovation management. Our revised propositions provide some management guidance in this respect. The central management takeaways are as follows: \medskip

Managers who allow participants greater autonomy can expect more positive tacit knowledge-sharing behaviours in open innovation partnerships. \citep{gagne2005self,gagne2009model,longo2017struggling}. Facilitating the sharing of know-how enables others to learn the practice that entails the know-how \citep{cook1999bridging}. Limiting autonomy through overly top-down or prescriptive management may result in sub-optimal open innovation outcomes. \medskip

Managers should identify tangible and intangible boundaries that may potentially affect knowledge sharing upfront. They can then put into place processes for overcoming these (e.g. cultural awareness training, employing language interpreters, planning more face-to-face meetings, arranging social events so that participants can get to know each other better, and so forth). Managers must guard against participants who put their interests ahead of the partnership or are not good team players. At the same time, managers must accept that openness is likely to vary among open innovation partners \citep{dahlander2010open}. Managers must also be sensitive to power imbalances in open innovation partnerships. Poor teamwork, lack of openness, and power imbalances can negatively impact open innovation by distorting or inhibiting tacit knowledge-sharing relations.  \medskip

Managers need to understand as well as explain how participants or partners will benefit from exchanging tacit knowledge. Less open partners may engage in some minor gatekeeping, but they may practise tertius gaudens brokerage in more extreme situations that can break trust \citep{obstfeld2014brokerage}. Focusing on relationship building will help build trust and contribute to greater openness. Managers must also ensure that everybody commits to a shared goal. Having partners commit to a shared goal contributes to a greater sense of camaraderie and helps build trust. One can expect better open innovation outcomes when partners are equally committed. There is merit in encouraging tertius inungens brokerage from the outset. Apart from facilitating and strengthening new connections, tertius inungens brokerage can help balance power relations and mediate a shared goal \citep{chesbrough2012open}.\medskip

Ultimately, the job of the open innovation manager is to find the right balance between structure and human agency. Too much structure may inhibit individual willingness to share tacit knowledge or contribute ideas. Conversely, too little structure can make goals less clear, leading to unsatisfactory open innovation outcomes. The key to finding the right balance is to get open innovation partners to commit to a common objective and allow human agency to thrive. Agency is what drives tacit knowledge sharing that is so critical for good innovation outcomes.

 \section{Study limitations}

Each of the partnerships investigated in this thesis was tackling a different innovation challenge. Given the differences and the cross-sectional nature of this study, one must be careful comparing cases. Generalisations will be limited and contingent on context \citep{welch2011theorising}. For example, we expect brokerage to feature more strongly in the early stages of open innovation, and we see this in our ERGM results. However, we cannot infer too much about the transient nature of brokerage from our cross-sectional data. Understanding how brokerage plays out over time would require the analysis of longitudinal data. \medskip

Tacit knowledge is hard to quantify, and the criteria used to determine how much of the knowledge was tacit in terms of codifiability, observability, and complexity are basic. Survey respondents are likely to have interpreted these criteria differently. One can expect the predominantly explicit and predominantly tacit knowledge provider networks to have some internal statistical variation. \medskip

The semi-structured interview questions were formulated while the quantitative data was still being collected. In hindsight, it would have been preferable to develop the semi-structured interview questions after completing the exponential random graph modelling. That way, the questions would have been more focused on seeking explanations to observed patterns of social interaction. Moreover, the semi-structured interviews covered only a fraction of the participants who responded to the online survey(33\%, 32\% and 18\% of survey respondents for each case, respectively). The interviewees did provide valuable information about critical factors affecting tacit knowledge exchanges. However, it was impossible to interview participants from all partner organisations, resulting in an incomplete picture. Furthermore, both the online survey and semi-structured interview questions were in English. Two of the three cases involved participants who spoke English as a second language. The extent to which this affected their understanding of, and response to, questions is not known. \medskip

Most academics accept that qualitative analysis is both interpretive and subjective \citep{aspers2019qualitative}. Each researcher brings their unique perspective when identifying and developing concepts and insights through close examination and reflection of their qualitative data. While the insights gained from the qualitative analysis helped the author make sense of his network data, he recognises that others may draw slightly different conclusions from what he did. Despite the author's best efforts, other researchers are likely to see things in the qualitative data he might have missed. Such is the reality of qualitative analysis. \medskip

\section{Future research}

Future studies might consider a longitudinal network study of knowledge-sharing behaviour across multiple open innovation projects. A longitudinal study will facilitate a more rigorous cross-case analysis. More specifically, it will shed light on the transient nature of brokerage and the emergence of informal structures deemed important for innovation. This study does suggest that connecting communities of practice may be important for open innovation. However, we did not see evidence of reciprocal sharing of tacit knowledge between partner organisations. How different communities of practice work across organisational boundaries is worthy of more attention. There is also a need to investigate more robust measures of tacit knowledge. This thesis broke new  ground in applying a critical realist approach to mixed-method social network analysis. Future studies should try an build upon the critical realist approach used in this thesis.
% Thank you and goodbye. Now fuck off.