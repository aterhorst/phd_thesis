% this is chapter 9
\section{Introduction}

This final chapter presents the main findings of this study and summarises what are considered to be important theoretical, methodological, and practical research contributions. It reflects on key lessons the author took on board during his PhD journey. The chapter also addresses study limitations and suggests possible avenues for future research. 

\section{Main findings}

% An open innovation project may be seen as a coordination mechanism for integrating specialist knowledge and competences of partner organisations. This study aims to improve the effectiveness of the management of open innovation projects by increasing our understanding of how tacit knowledge is shared when the locus of innovation resides in a network of social relations that span different locations and organisational boundaries. \medskip

% This study successfully showed how ERGMs can be used to explore knowledge-sharing behaviour in three open innovation partnerships. The modelling highlights the differences in network substructures and actor–relation effects according to how much of the knowledge is tacit in nature. \medskip

% Autonomous motivation of individuals determines the extent to which they share their tacit knowledge and receive tacit knowledge from others. Brokerage and hierarchy effects were significant in both open innovation projects. It is argued that project managers need to promote strong brokerage in the early stages of an open innovation project. Brokers play a key role by connecting people in different organisations and helping people make sense of unfamiliar knowledge. Once people get to know each other and start sharing tacit knowledge on their own volition, brokers are no longer needed as much. Project managers might beneficially facilitate tacit knowledge exchange by creating opportunities for face-to-face interaction. They might also identify the need to shape a culture that encourages autonomy, competence, and relatedness in open innovation projects as this will contribute to more positive tacit knowledge-sharing behaviour. \medskip

% Ultimately, the job of the project manager is to strengthen ties that are initially weak in their open innovation knowledge networks so that knowledge is exchanged, new knowledge is created, and things get done. \medskip

\section{Research contribution}

This study has advanced management theory, promoted a more rigorous approach to mixed-method social network analysis, and suggested better ways to manage tacit knowledge flows in open innovation.

\subsection{Theoretical contribution}

This study contributes to knowledge management theory by explaining how different types of brokerage moderate knowledge acquisition, assimilation, transformation, and exploitation processes in open innovation.

\subsection{Methodological contribution}

% CR - mixing with rigour

\subsection{Practice contribution}

% - Skilled brokerage in open innovation
% - Recognising boundaries in multidisciplinary projects
% - Organisational behaviour - creating shared goals/understanding



\section{Study limitations}

This study has several limitations. The study was cross-sectional in nature and we could only infer the evolutionary aspects of open innovation through cross-case analysis. Even then, one had to be circumspect, having studied only three open innovation projects. The online survey was conducted in English yet, for some respondents, English is a second language. How this affected their understanding of, and response to, survey questions is not known. \medskip

Tacit knowledge is hard to quantify and the criteria used to determine how much of the knowledge being provided was tacit in nature (codifiability, observability, and complexity) are basic. Survey respondents are likely to have interpreted these criteria differently and one can expect the predominantly explicit and predominantly tacit knowledge provider networks to have some internal statistical variation. \medskip

\section{Future research}

Future research might consider a longitudinal network study of knowledge-sharing behaviour across multiple open innovation projects. Not only will this facilitate more thorough cross-case analysis, it will also shed light on the transient nature of brokerage and emergence of informal structures deemed important for innovation. There is also a need to investigate more robust measures of tacit knowledge. \medskip
