\section{Introduction}


Before continuing, it may be timely to remind readers of the research questions addressed in this thesis:

\begin{itemize}
\item What does the structure of tacit knowledge networks reveal about knowledge enacted in practice?
\item Does brokerage differ according to the type of knowledge being exchanged?
\item To what extent does self-determination drive tacit knowledge sharing in open innovation?
\item What does the micro-structure of tacit knowledge networks reveal about trust and power-relations in open innovation partnerships?
\end{itemize}

- narrative => power
- communities of practice => power
- 

% Levesque, V. R., Calhoun, A. J., Bell, K. P., & Johnson, T. R. (2017). Turning contention into collaboration: engaging power, trust, and learning in collaborative networks. Society & Natural Resources, 30(2), 245-260.

% Balland, P. A., Boschma, R., & Frenken, K. (2015). Proximity and innovation: From statics to dynamics. Regional Studies, 49(6), 907-920.

This chapter uses the logic of retrodiction to pull the results of the statistical social network analysis (Chapter 6) and the results of the qualitative analysis of semi-structured interviews (Chapter 7) together. Retrodiction is a process which uses past observations, events and data as evidence to infer the mechanisms that produced them \citep{wynn2012principles}. In this instance, we apply the logic of retrodiction to update or refine the initial set of seven propositions using evidence from the three case studies. Essentially, we transform the propositions into more general statements about potential mechanisms for tacit knowledge sharing in open innovation. From a critical realist perspective, the more general statements represent the real (i.e. knowledge of what and why things are).

% Recruiting appropriate open innovation partners is a challenge. Ideally, selection criteria should be based on the technical competence and the complementarity of partner resources. Cultural alignment, adaptability, long-term strategic orientation, organisational capabilities, economic performance, financial stability, mutual awareness of each other's capabilities and technological transparency also need to be considered \citep{aloini2016structured}.\medskip

% People who are both strong knowledge producers and great collaborators enhance their firm’s innovative output \citep{grigoriou2014structural}. Relational stars possess strong combination of both human and social capital, that is, a strong individual-level productivity performance combined with a highly consequential network position \citep{grigoriou2014structural}.\medskip

% \subsection{Overhead}

% Though connectivity and teamwork are considered key to organisational success, this comes at a cost. The communication overhead of collaboration can become a significant burden \citep{cross2013has}. Recent studies show that the time spent by people engaged in collaborative activities has grown by 50\% over the past two decades. Employees spend up to 80\% of their time in meetings or answering colleagues’ requests, leaving little time for doing critical work on their own \citep{cross2013has,cross2015investing,cross2016collaborative}. 

% \citet{cross2016collaborative} distinguish three types of collaborative resources: "informational resources" are knowledge and skills that can be recorded or passed on, "social resources" involving one's awareness, access and position in the network, which can be used facilitate collaboration, and "personal resources" that include one's own skills, time, and energy. Whereas informational and social resources are more easily shared, there is a finite amount of time a person can contribute to any collaborative activity \citep{cross2016collaborative}. Burn-out, disengagement can lead to under-performing networks \citep{cross2013has}.

% This is evident in Cases 1 and 3 ...

% Leaders can solve this problem in two ways: by streamlining and redistributing responsibilities for collaboration and by rewarding effective contributions \citep{cross2016collaborative}. 

% Boundary-spanning networks are semi-formal in the sense these still need to be properly managed \citep{cross2015investing}. These networks need to be carefully cultivated. This may include reconfiguring such networks over time. Once networks are established, mechanisms are needed to track and measure both network performance and collaboration outcomes. It is important to identify where connectivity can produce value, rather than shooting blindly with interventions that simply layer on collaborative demands \citep{cross2015investing}.

% Although open innovation has been widely adopted, the concept is still evolving. Most studies have emphasised the upsides of open innovation and paid little attention to the downsides \citep{hossain2013open}. Past studies have tended to focus on open innovation involving large multinational firms. 

% \section{Autonomous motivation predicts tacit knowledge sharing} 

% Our second proposition stated that actors with higher levels of autonomous motivation are more likely to share their tacit knowledge with, or seek tacit knowledge from, others. Autonomous motivation means engaging in an activity because it is either enjoyable or personally meaningful \citep{gagne2005self}. Autonomy involves acting with a sense of volition. Since sharing tacit knowledge is largely an act of volition \citep{polanyi1966tacit}, one would expect autonomous motivation to be a reliable predictor of tacit knowledge sharing. All three cases have a significant and positive autonomous motivation receiver effect for tacit knowledge sharing. People seeking out and receiving tacit knowledge in Projects 1 and 2 are autonomously motivated to do so. The receiver effect makes sense, given respondents were asked to name people who provided them with useful knowledge. Posing the question this way emphasised knowledge seeking rather than knowledge-providing behaviour. Interestingly, neither controlled nor autonomous motivation is a significant factor as far as sharing predominantly explicit knowledge is concerned. The modelling results suggest that knowledge sharing is not something that can be incentivised in extrinsic ways, which is consistent with findings of past studies \citep{bock2001breaking, bock2005behavioral,hung2011influence}. Our modelling results indicate that actors with higher levels of autonomous motivation are more likely to share their tacit knowledge with, or seek tacit knowledge from, others. In other words, our second proposition is supported.

% \section{Management implications}

% This study is concerned with open innovation projects and knowledge-sharing behaviours in open innovation projects. A key question is the extent to which these findings can be of broader relevance to open innovation management. \medskip

% Results from this study show that autonomous motivation predicts tacit knowledge sharing. According to \citet{gagne2009model}, managerial styles that promote autonomy, competence, and relatedness encourage knowledge sharing (autonomy referring to the level of volition in self-determination behaviour). Managers who foster appropriate cultures to support the autonomous motivation of individuals participating in open innovation projects can expect more positive knowledge-sharing behaviours \citep{gagne2005self,gagne2009model}. \medskip

% Open innovation implies multiple organisations in different locations. The degree of tacit knowledge transfer depends on the closeness of partners. Frequent interactions afford the two parties the ability to understand each other’s needs and satisfy the needs accordingly \citep{cavusgil2003tacit}. This study confirmed that closeness does facilitate tacit knowledge exchange. Managers may benefit from understanding the importance of closeness in achieving a shared understanding of project aims, the challenges involved, and opportunities for emergent solutions. \citet{constant1994s} found that individuals with higher expertise were more likely to share useful knowledge when other employees asked questions. Encouraging face-to-face interactions could help satisfy people’s innate need for relatedness and create opportunities for the formation of social relationships that facilitate tacit knowledge exchange \citep{eckblad2016organizing}. \medskip

% This study also highlights the significance of brokerage. Brokers play a key role in the early stages of open innovation. Not only do they connect otherwise disconnected groups of people, they also facilitate the translation, integration, and combination of diverse knowledge \citep{davis2010agency}. Managers need to encourage brokerage in the early stages of an open innovation project, as brokerage is an important mechanism for building stronger informal social structures or relationships. As these informal social relationships become established, the need for brokerage diminishes \citep{burt2004structural, obstfeld2014brokerage}. Essentially, the task of the manager is to foster the conversion of weak ties into strong ties in order to get things done \citep{burt2004structural,phelps2012knowledge,rost2011strength,steen2011small}. This study shows that the type of knowledge being brokered may depend upon how radical a particular innovation is, and where more radical outcomes are being sought, then brokerage of tacit knowledge is likely to be particularly important. \medskip

% Importantly, in using ERGMs to analyse the data we have an analytic and statistical framework in which to unpack brokerage and closure mechanisms in ways that other network methods cannot. \medskip



Evasive knowledge hiding in academia: When competitive individuals are asked to collaborate \citep{hernaus2019evasive}.