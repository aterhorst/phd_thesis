% literature review

\section{Introduction}

In the previous chapter, an open innovation partnership was conceptualised as a knowledge network constituted to achieve a specific innovation outcome. The chapter introduced social network analysis and explained how it can be used to test hypotheses or propositions about network tie formation that relate to psychosocial theory. It then went on to describe important social mechanisms underpinning knowledge networks and how these facilitate innovation. \medskip

Persuading people to share their knowledge is an important part of knowledge management. Unless a person or a group decides to share their knowledge, it remains a hidden and untapped resource \citep{davenport1998working}. Knowledge sharing can be understood as the behaviour by which an individual or group voluntarily gives others access to their unique knowledge and experiences \citep{cabrera2002knowledge,hansen2005share}. This chapter looks at how motivation, trust, and power influence knowledge sharing behaviour and what this means in terms of the \enquote{agency versus structure} debate. Structure refers to the patterned arrangements that influence or limit the choices and opportunities available to individuals \citep{bandura1999social}. Agency, on the other hand, is the capacity of individuals to act independently and make their own free choices, and thus lies at the heart of tacit knowledge exchange \citep{leonard1998role,barker2016cultural}. \medskip

Past studies reveal innovation mostly happens through through informal structures \citep{allen1977role,ibarra1993network}. Agency drives the emergence of informal structures and managers need to structure open innovation partnerships in ways that allow participants sufficient agency to forge their own relationships. Too much structure may inhibit people's willingness to share and co-create knowledge. Conversely, too little structure can make goals less clear, leading to unsatisfactory open innovation outcomes. The extent to which structure constrains or encourages agency in open innovation has not received much attention in the literature thus far. \medskip

\section{Motivation}

Communicating tacit knowledge requires significant individual effort \citep{linde2001narrative,desouza2003facilitating}. People must be sufficiently motivated to seek out and share tacit knowledge \citep{leonard1998role}. Motivation is a theoretical construct used to explain individual behaviour. A motive is what prompts a person to act in a specific way or develop an inclination towards a particular behaviour \citep{pardee1990motivation}. Understanding how people are motivated to share their tacit knowledge in open innovation partnerships is central to this study. \medskip

\subsection{Achieving greater self-efficacy}

According to \citet{bandura1989human}, motivation lies at the heart of human agency. People exercise agency through self-belief of efficacy, i.e. through their belief in their capability to organise and execute the courses of action required to manage prospective situations \citep{white1959motivation, bandura1994self}. Humans learn much by observing the behaviour of others and assessing what the consequences of such behaviour is for them \citep{bandura1999social}. By observing others, an individual can see how patterns of behaviour are performed, which then serves as a guide for future action. Feedback after taking action helps the individual correct or adjust their behaviour and thus develop a stronger sense of self-efficacy \citep{bandura1977self}. This suggests people will be more inclined to seek out or share tacit knowledge if this is perceived as normal behaviour. Because self–efficacy is the extent to which an individual believes that they can master a particular skill, people with high self-efficacy are more likely to engage in learning to develop their know-how or expertise \citep{bandura1986social,gist1989influence,zimmerman2000self}. Challenges often stimulate people with high self-efficacy to greater efforts, whereas someone with low self-efficacy will tend toward discouragement and giving up \citep{gist1992self,zimmerman1992self}. From this, we can postulate: \medskip

\begin{tcolorbox}
\emph{Proposition 1: People with higher-levels of self-efficacy are more likely to seek out tacit knowledge and generate new ideas.}
\end{tcolorbox}

\subsection{Intention to share knowledge}

The theory of planned behaviour suggests actions and behaviours reliably follow intentions. Intentions capture the motivational factors that influence a particular behaviour \citep{ajzen1985intentions}. The theory of planned behaviour postulates three conceptually independent determinants of intention: First, is attitude toward the behaviour and refers to the degree to which a person has a favourable or unfavourable disposition towards the behaviour in question. Second, is subjective norm, which refers to the perceived social pressure to perform or not perform the behaviour. Third, is the degree of perceived self-efficacy, which as stated earlier, refers to how people rate their ability to succeed in specific situations or accomplish a task \citep{bandura1977self,bandura1982self,ajzen1991theory}. From this, one can argue that the motivation to share tacit knowledge is not just dependent on perceived self-efficacy, but also on personal attitudes and subjective norms \citep{gagne2009model,chen2012behavioral}. \medskip

Perceived social pressure to perform or not to perform the behaviour in an organisational setting stems from the prevailing workplace culture \citep{papadopoulos2013exploring}. \citet{cooke1988behavioral} distinguishes between constructive and defensive workplace cultures. Constructive cultures emphasise excellence, self-efficacy, autonomy, and teamwork. Defensive cultures tend to be passive, where organisational members feel pressurised to think and behave in ways that are inconsistent with their personal beliefs, or aggressive, where the emphasis is on individual achievement or competitive behaviour \citep{cooke1988behavioral}. Hence,  \medskip

\begin{tcolorbox}
\emph{Proposition 2: People are more likely share or seek out tacit knowledge in a constructive cultural setting.}
\end{tcolorbox}

\subsection{Level of self-determination}

Self-determination theory suggests humans are innately driven to seek out competence (personal self-efficacy), autonomy (rights to agency) and relatedness \citep{ryan2000self}. The theory distinguishes between controlled and autonomous motivation. Being controlled involves acting under some sense of pressure. Autonomy involves acting with a sense of volition and having the experience of choice. Intrinsic motivation, driven by an individual’s interest or pleasure in the activity itself, is essentially autonomous. Less interesting activities require extrinsic motivation, where people's actions depend on the perceived contingency between behaviour and the desire for implicit approval or tangible rewards \citep{gagne2005self}. Extrinsic motivation can vary in the degree to which it is autonomous or controlled. Externally regulated behaviour is a form of controlled motivation. Regulation that has been taken in by a person but is not accepted as their own is said to be introjected and may be seen as a form of internalised extrinsic motivation. With identified regulation, people feel a greater sense of autonomy because their behaviour is aligned to their personal goals and identities \citep{gagne2005self}. \medskip

Past studies show autonomous motivation facilitates effective performance and well-being, whereas controlled motivation can detract from those outcomes, particularly if the task requires creativity, cognitive flexibility, or deep processing of information \citep{gagne2005self}. \citet{gagne2009model} presents a process model of knowledge-sharing motivation based on the theory of planned behaviour and self-determination theory. The model assumes autonomous motivation predicts knowledge sharing intention, which in turn predicts knowledge-sharing behaviour \citep{gagne2009model}. The more internalised the individual’s motivation to share knowledge, the more likely knowledge sharing will result \citep{witherspoon2013antecedents}. As long as knowledge sharing is voluntary, sharing parties tend to perceive it as intrinsically satisfying because it is self-determined and enhances their sense of self-worth and competence \citep{kaser2001knowledge,lam2010knowledge,dumbach2014establishing}. Because tacit knowledge sharing is a social act, it also satisfies an individual's need for relatedness \citep{llopis2016understanding}. People who are autonomously motivated are thus more likely to build their know-how or expertise (i.e. tacit knowledge) compared to those inclined towards controlled motivation. Hence, \medskip
\begin{tcolorbox}
\emph{Proposition 3: People with higher levels of autonomous motivation are more likely to seek out tacit knowledge.}
\end{tcolorbox}


\section{Trust}

Past studies show that trust lies at the heart of knowledge exchange \citep[e.g.][]{nonaka1994dynamic,davenport1998working,hsu2007knowledge,wang2016examining}. Interpersonal trust may be defined as \enquote{the willingness of a party to be vulnerable to the actions of another party based on the expectation that the other will perform a particular action important to the trustor, irrespective of the ability to monitor or control that other part} \citep{mayer1995integrative}. Trust is cognition-based insofar as \enquote{we choose whom we will trust in which respects and under what circumstances, and we base the choice on what we take to be \enquote{good reasons}, constituting evidence of trustworthiness} \citep{lewis1985trust}. Trust limits opportunistic behaviour, reduces uncertainty, builds commitment, facilitates cooperation, and creates a safe environment in which to resolve issues \citep{nonaka1994dynamic,panteli2005trust,rasmussen2007work}. Emotional ties linking individuals provide the basis for affect-based trust \citep{mcallister1995affect}. Affect-based trust has a significantly greater effect on willingness to \emph{share} tacit knowledge, whereas cognition-based trust plays a greater role in willingness to \emph{use} tacit knowledge \citep{levin2004strength,holste2010trust,ding2015research}. \medskip

\begin{tcolorbox}
\emph{Proposition 4a: People are more likely to share their tacit knowledge with others they trust at an emotional level.} \\
\emph{Proposition 4b: People are more likely to use tacit knowledge provided by others they perceive to be competent.}
\end{tcolorbox}

\subsection{Swift trust}

Trust is usually seen as something that develops gradually over time through observation of past behaviour \citep{mayer1995integrative}. Achievement of trust at one level enables the development of trust at the next level and so on \citep{robert2009individual}. \enquote{Swift trust}, on the other hand, is a presumptive trust that may explain trusting behaviour exhibited by members of \enquote{temporary organisations} \footnote{Temporary organisations are encountered in a vast range of social and economic activities and across a range of industries. In the commercial sector, they may involve a joint effort to develop a new technology or product, bring about organisational renewal, or enter a new market \citep{janowicz2009research}. An open innovation partnership can be described as a temporary organisation created to tackle a specific innovation challenge \citep{cococcioni2014exploring}.}. With swift trust, trust is initially assumed by members of a temporary organisation (where members may be individuals or groups). Influenced by their own disposition towards trust, members use \enquote{category-based information processing} to form an initial swift trust judgement. This involves mentally placing others into a category and using this as a basis of trust. Once members have acquired some information about other team members' behaviour, they use this knowledge about their ability, integrity, and benevolence to form a cognition-based trust judgement. Adjustments are made to trust beliefs on the basis of fulfilled or unfulfilled expectations \citep{meyerson1996swift,robert2009individual}. Swift trust is particularly important in temporary organisations because it substitutes the traditional mechanisms of control and coordination found in established hierarchies \citep{kasper2001communicating}. \medskip

\begin{tcolorbox}
\emph{Proposition 5: Swift trust is likely to feature strongly in open innovation partnerships.}
\end{tcolorbox}

\subsection{Inter-organisational trust}

Whereas interpersonal trust refers to the positive expectation of someone's behaviour, inter-organisational trust is more about confidence in an organisation's reliability and integrity \citep{zaheer1998does,ashnai2016inter}. Interpersonal trust feeds inter-organisational trust. Higher levels of inter-organisational trust can help reduce conflict, minimise transaction costs, and increase the likelihood that knowledge acquired from others is sufficiently understood and absorbed \citep{zaheer1998does,abrams2003nurturing,levin2004strength}. Hence, \medskip

\begin{tcolorbox}
\emph{Proposition 6: People are more likely to share their tacit knowledge when inter-organisational trust is high (indicated by the overall level of inter-personal trust)}.
\end{tcolorbox}

\subsection{Reciprocal nature of trust}

Trust in knowledge sharing networks may be indicated by reciprocity or transitive closure. Reciprocity is a key concept in social exchange and game theory and reflects a human tendency to return helpful or harmful acts in kind \citep{nowak2005evolution}. Reciprocity helps build trust and strengthen ties between actors \citep{blau1964exchange,axelrod1984evolution}. \medskip 

Social exchange theory views exchange as a social behaviour that may result in both economic and social returns. People act to maximise the potential benefit of economic and social returns, where one person does another a favour with a general expectation of some future return, even though the exact nature of this return is not stipulated in advance. Actors consciously or sub-consciously use cost-benefit analysis to optimise their social interactions so that the potential rewards outweigh any costs. Reciprocation is less likely if costs are too high \citep{homans1961social,blau1964exchange}. Exchanges can be characterised in terms of behavioural or relational responses. A positive initiating action would increase trust, a type of relational response, which in turn would promote a positive behavioural response \citep{cropanzano2016social}. Trust is more likely to develop between partners when exchange occurs without explicit negotiations or binding agreements. Under such conditions, the risk and uncertainty of exchange provide the opportunity for partners to demonstrate their trustworthiness  \citep{molm2000risk}. Inadequate reinforcement or asymmetry in economic or social returns may result in the breakdown of trust \citep{homans1961social}. \medskip

In game theory, the \enquote{prisoner’s dilemma} is a game that explores the tension between individual self-interest and collective cooperation \citep{richards2001reciprocity}. This is particularly salient with respect to open innovation. Economic theory suggests individuals or groups are primarily motivated by self-interest \citep{axelrod1984evolution}. This implies people will not be inclined to share knowledge unless they see some advantage or payoff accruing to themselves \citep{yang2006knowledge}. Modelling of the \enquote{prisoner's dilemma} show people can maximise their pay-off through cooperative action based on simple reciprocity or \enquote{tit-for-tat} behaviour \citep{axelrod1981evolution,richards2001reciprocity}. Tit-for-tat involves cooperating with your partner on the first round, then adjusting your behaviour to match your partner’s in subsequent rounds. If your partner cooperates in a reciprocal manner, you continue to cooperate. However, should they defect, you respond in kind by retaliating immediately against them. The assumption is that people who start out with good intentions (e.g. by being nice, generous, or forgiving) are more likely to be cooperative. One can argue that a nice tit-for-tat strategy is the basis of swift trust \citep{fulmer2013trust,mollering2013process}. Fundamental to tit-for-tat trust is the notion that trust decisions are rational and calculative \citep{fulmer2013trust}.

\begin{tcolorbox}
\emph{Proposition 7: Reciprocity in tacit knowledge sharing indicates higher levels of trust in open innovation partnerships.}
\end{tcolorbox}

\section{Power}

 Though it is often said that \enquote{knowledge is power}, the relationship between knowledge and power is complex and not well understood. Powerful actors can control the production and dissemination of knowledge as a way of shaping agendas \citep{gaventa2007power}. Knowledge itself is also a source of power, as knowledgeable people are more persuasive and thus able to influence agendas \citep{hart1997power}. Two contrasting views of power dominate the power literature: \enquote{power as domination}, also referred to as \enquote{power-over}, and \enquote{power as empowerment}, often characterised as \enquote{power-to} \citep{haugaard2012rethinking}. \medskip
 
\subsection{Power as a relational construct}

Both power-over and power-to treat power as a relational construct, where the distribution of power reflects the structure of exchange opportunities \citep{blau1964exchange,reagans2008knowledge,bonacich2009structural}. The power of one actor over another reflects the dependence of the second on the first for a valued resource \citep{emerson1962power}. Power imbalances may lead to those feeling disempowered or powerless to engage in cost-reduction or balancing operations \citep{emerson1962power}. Cost reduction is a process in which a person adjusts his or her personal values to accommodate the demands of a powerful other. Essentially, the less powerful person submits to the more powerful other. This behaviour does not alter the balance of power. Balancing operations, on the other hand, aim to shift the balance of power. For example, tensions generated by power inequality can result in network extension where power-disadvantaged actors may seek out alternate relations to reduce their dependence on a given actor for valued resources \citep{cook2013social}. Other ways to balance power include motivational withdrawal or deference \citep{emerson1962power}. Power-to can also be treated as a relational construct. Powerful actors can empower others through power sharing or delegating authority. Since powerful actors control who they share power with or delegate authority to, one can argue this is a very limited form of empowerment \citep{conger1988empowerment}. \medskip

\citet{todorova2007absorptive} suggest internal power relationships moderate knowledge assimilation and transformation processes, while external power relationships moderate the relationship between absorptive capacity and competitive advantage. \citet{easterby2008absorptive} investigate knowledge processes at both the intra- and inter-organisational level. They distinguish between power involving discrete political acts by self-interested actors, which they refer to as \enquote{episodic power}, and power which is diffused through social structures, referred to as \enquote{systemic power}. \citet{easterby2008absorptive} find external access to knowledge is affected by episodic power, while the diffusion of knowledge within a firm depends mainly on systemic power. \medskip

This suggests that actors who do not award status to more powerful actors are also likely to withhold their tacit knowledge from them. However, actors may be more willing to seek out tacit knowledge from more central actors whose power stems from deference relating to their expertise. Alternatively, actors more likely to share tacit knowledge with others at a similar level of power as them. \medskip

\subsection{Power as a motivational construct}

Empowerment can be treated as a motivational construct if it satisfies an intrinsic need for self-determination or self-efficacy \citep{deci1989self,bandura1986social}. The need for power is satisfied when actors believe they are able to cope with the physical and social demands of the environment \citep{conger1988empowerment}. Empowerment is not limited to individuals. From an organisational perspective, empowerment may be conceptualised as a \enquote{process of enhancing feelings of self-efficacy among organisational members through the identification of conditions that foster powerlessness and through their removal by both formal organisational practices and informal techniques of providing efficacy information}. Any strategy that weakens the self-determination need or self-efficacy belief of organisational members will increase their feelings of powerlessness \citep{conger1988empowerment}. This speaks directly to the \enquote{agency versus structure} debate, i.e. power-to promotes personal agency whereas structure is all about exercising power-over others.  \medskip

\begin{tcolorbox}
\emph{Proposition 8: Too much structure in open innovation partnership inhibits tacit knowledge sharing.}
\end{tcolorbox}

\subsection{Power as discourse}

Power is produced when people negotiate meaning in context of existing power and knowledge relations \citep{foucault1980power,rouse2005power,heizmann2015power}. 


\subsection{Relationship between trust and power}

An actor may want to demonstrate their trustworthiness to maintain a specific relationship \citep{schilke2015power}. Deciding to trust someone or not depends on how valuable an one perceives the relationship with the other person to be \citep{axelrod1984evolution}. In a power-dependence relation, a powerful actor can reasonably expect the less powerful other to place a high value on their exchange relationship because they do not have many options to access this resource elsewhere and are thus more dependent on the more powerful actor for the resource \citep{emerson1962power}. Given the dependence of the less powerful exchange partner, the more powerful party is more likely to trust their less powerful partner. A less powerful actor would expect the power-advantaged actor to have several other valuable exchange opportunities and be less dependent on any particular relationship. This would suggest the power disadvantaged actor will be less trusting of more powerful others \citep{schilke2015power}. Contrary to this expectation, \citet{schilke2015power} find that less powerful actors tend to be more trusting than powerful actors. They suggest with increasing dependence, low-power individuals invest more cognitive resources in processing trust-related information that becomes available in an exchange than high-power individuals do. Less powerful actors are motivated to see the more powerful other as trustworthy to avoid anxiety stemming from their feelings of dependence. Hope and perceived benevolence feature strongly in the trust decisions of power-disadvantaged \citep{schilke2015power}. \medskip

Tacit knowledge is a tremendous source of personal power and deference. Resisting the formalisation of tacit knowledge can help maintain personal power \citep{schultze2004knowing}. While a less powerful actor might be less inclined to share their tacit knowledge with a more powerful other, they may end up doing so anyway to reduce anxiety. A more powerful actor may be less inclined to trust others with their tacit knowledge. \medskip

\emph{Proposition 7: Open innovation partners who perceive themselves to be power-disadvantaged are less likely to trust other more powerful partners.}

\emph{Proposition 8: Open innovation partners who perceive themselves to be power-disadvantaged are more trusting than other more powerful partners.}

In other words, preferential attachment suggests that some people in a knowledge exchange network will acquire knowledge at a faster rate than others. Given that knowledge is a potential source of power, this has implications for power-dependence relations. More central actors are able to exert much more influence because they have greater access to diverse knowledge \citep{emerson1962power,bonacich1987power}. 

%Potential power and power use: An investigation of structure and behavior \citep{brass1993potential}
%Power decreases trust in social exchange \citep{schilke2015power}.
%Structural Implications of Reciprocal Exchange: A Power- Dependence Approach \citep{bonacich2009structural}
%Network knowledge and the use of power \citep{simpson2011network}

\section{Brokerage}
% Fleming and Wag...2007
Brokerage may be conceptualised as a social process that shapes interaction between two or more parties in a wide variety of triadic structures \citep{obstfeld2002knowledge}. Much of the organisational literature dealing with social structures considers brokerage in terms of an open triad, one in which a broker has ties to two alters who are not tied to one another. Indeed, \citet{marsden1982brokerage} defines brokerage as a mechanism \enquote{by which intermediary actors facilitate transactions between other actors lacking access or trust in one another}. Likewise, \citet{fernandez1994dilemma} describe brokerage as a \enquote{relation in which one actor mediates the flow of resources or information between two other actors who are not directly linked}. While brokers may be successful in getting others to appreciate their good ideas, implementing those ideas is another story \citep{marabelli2016}.\medskip

\citet{obstfeld2014brokerage} describe three strategic orientations to brokerage: \enquote{conduit brokerage} is about a third-party who transfers information, knowledge, or other resources between two disconnected parties. The broker mediates rather than moderates the relationship between two others and may help them synthesise new knowledge. With \enquote{tertius gaudens} brokerage, the broker aims to exploit differences between two parties by either keeping them apart or playing one against another. Moreover, the broker who stand between unconnected alters also benefit from the novel information that such a structure affords \citep{burt1992structural}. \enquote{Tertius gaudens} brokerage is more about empowerment of self than about empowering others. In contrast, \enquote{tertius iungens} brokerage is about a third-party who introduces two otherwise disconnected parties to each other and encourages them to collaborate (Table \ref{brokerage}). \medskip 

With \emph{tertius iungens} brokerage, a third party introduces or facilitates interaction between two other parties. While \emph{tertius gaudens} exploits disconnection or negative ties, \emph{teritius iungens} actively pursues coordination. \emph{Tertius iungens} brokerage is most opportune when the broker detects opportunities to connect complementary, rather than redundant, alter attributes such as resources and abilities \citep{obstfeld2014brokerage}. Both conduit and iungens draw on knowledge articulation, or the social process by which knowledge is made more explicit, useful, or relevant to the situation at hand \citep{obstfeld2005social,obstfeld2011saying,obstfeld2012creative}.

\citet{cohen1990absorptive} see brokers with diverse expertise and contacts playing a key role in the acquisition of knowledge. They believe a diversity of internal perspectives is important to counter resistance towards external knowledge or ideas. From an open innovation perspective, \emph{tertius iungens} brokerage is quite important, especially in the early stages of collaboration, when many actors do not know fellow collaborators in partner organisations. Not only that, knowledge held by third parties is also likely to be unfamiliar. Once ties have been established through \emph{tertius iungens} brokerage, \emph{conduit} brokerage is needed to sustain swift trust and help synthesise and transform knowledge into novel ideas \citep{fleming2007collaborative,lingo2010nexus,quintane2016brokers}. \citet{gould1989structures} describe \emph{conduit} brokerage in terms of \emph{coordination}, \emph{liaison}, \emph{itinerant}, \emph{representative}, and \emph{gatekeeper} roles (Table \ref{gf_params}). Each role expresses a different power dynamic and one should be able to use the relative proportion of each role to characterise power-relations in open innovation collaborations. \medskip

\citet{tortoriello2010activating} investigate the paradox in which diversity of knowledge and information available across organisational boundaries is necessary to spur innovation, yet at the same time may be a barrier for successful knowledge sharing and integration. Advantages stemming from bridging ties are contingent upon the configuration or microstructure of the ties forming the bridge \citep{tortoriello2010activating,tortoriello2015social}. Common third-party ties facilitate shared understanding, reduce friction due to differences in understanding, and promote cooperation and coordinated actions that are necessary to integrate and take advantage of diverse sources of knowledge \citep{tortoriello2010activating}. Such brokerage is a crucial means by which intra- and inter-organisational networks evolve, expand and drive change \citep{obstfeld2014brokerage}.  \medskip

brokerage in the  tertius gaudens  tradition inhibited “mutual ownership and understanding of the new [idea] combination,” and, though the idea was more novel, it was less likely to be reused than ideas arising from more cohesive networks \citep{marabelli2016}.

\section{Interdependence of personal agency and structure}

As mentioned before, personal agency operates within a network of social structural influences. Though such structures shape human practices, it is also human practices that shape or constitute social structures \citep{sewell1992theory}. Structures are enacted by \enquote{knowledgeable} human agents who know what they are doing and how to do it. Agents act by putting into practice their structured knowledge. This conception of human agents as both \enquote{knowledgeable} and \enquote{enabled} implies that agents are capable of putting their structurally formed capacities to work in creative or innovative ways \citep{giddens1984constitution}. \medskip

\citet{davis2010agency} explores agency in innovative inter-organisational relationships.






% \begin{table}[]
% \small
% \centering
% \caption{Three forms of brokerage process. Reproduced from \citet{obstfeld2014brokerage}.}
% \label{brokerage}
% \begin{tabularx}{\textwidth}{>{\raggedright}p{3.5cm}>{\raggedright}p{3.5cm}>{\raggedright}p{3.5cm}>{\raggedright}p{3.5cm}}
% 	\toprule
% 	& \multicolumn{1}{c}{Conduit} & \multicolumn{1}{c}{Tertius Gaudens} & \multicolumn{1}{c}{Tertius Iungens} \\ 
% 	\midrule
% 	& \begin{minipage}{.2\textwidth} \centering \includegraphics[width=0.7\linewidth]{Images/CDT_brokerage} \end{minipage}  & \begin{minipage}{.2\textwidth} \centering \includegraphics[width=0.7\linewidth]{Images/TG_brokerage_1} \end{minipage}  & \begin{minipage}{.2\textwidth} \centering \includegraphics[width=0.7\linewidth]{Images/TI_brokerage} \end{minipage}   \\
% 	&  &  & \begin{minipage}{.2\textwidth} \centering \includegraphics[width=0.7\linewidth]{Images/TG_brokerage_2} \end{minipage}  &  \\
% 	\midrule
% 	Open network\\(absence of A-C tie) & B transfers information, knowledge, or other resources between A and C, where A and C have no prospect of meeting. & B plays A and C against one another or keeps A and C apart. & B introduces A and C where A and C have no prior tie. &
% 	\midrule
% 	Closed Network\\(presence of A-C tie) & B facilitates transfer between A and C and may help synthesise new knowledge. & B cultivates conflict,competition, or separation between A and C (divide et impera) & B coordinates new collaborative action between A and C. & \\ 
% 	\bottomrule
% \end{tabularx}
% \end{table}





\begin{table}[]
	\small
	\centering
	\caption{Gould-Fernandez brokerage roles. After \citet{gould1989structures} and \citet{butts2016package}.}
	\label{gf_params}
	\begin{tabularx}{\textwidth}{@{}lcl@{}}
		\toprule
		\multicolumn{1}{c}{Role} & \multicolumn{1}{c}{Motif} & \multicolumn{1}{c}{Explanation} \\ \midrule
		Liaison (b\textsubscript{O})			&  \begin{minipage}{.2\textwidth} \centering \includegraphics[width=0.4\linewidth]{Images/b_O} \end{minipage}	& \begin{tabular}[c]{l}Broker mediates contact between two\\ individuals from different groups,\\ neither of which is the group to\\ which he or she belongs.\end{tabular}\\ [10ex]
		Representative  (b\textsubscript{IO})	& \begin{minipage}{.2\textwidth} \centering \includegraphics[width=0.4\linewidth]{Images/b_IO} \end{minipage}   & \begin{tabular}[c]{l}Broker mediates an outgoing contact\\ from an in-group member to an\\ out-group member.\end{tabular}\\ [10ex]
		Gatekeeper (b\textsubscript{OI})		& \begin{minipage}{.2\textwidth} \centering \includegraphics[width=0.4\linewidth]{Images/b_OI} \end{minipage}   & \begin{tabular}[c]{l}Broker mediates an incoming contact\\ from an out-group member to an\\ in-group member. \end{tabular}\\ [10ex]
		Itinerant broker (w\textsubscript{O})	&  \begin{minipage}{.2\textwidth} \centering \includegraphics[width=0.4\linewidth]{Images/w_O} \end{minipage}   & \begin{tabular}[c]{l}Broker mediates contact between two\\ individuals from a single group to\\ which he or she does not belong. \end{tabular}\\ [10ex]
		Coordination (w\textsubscript{I})		& \begin{minipage}{.2\textwidth} \centering \includegraphics[width=0.4\linewidth]{Images/w_I} \end{minipage}    & \begin{tabular}[c]{l}Broker mediates contact between two\\ individuals from his or her own\\ group. \end{tabular}\\ 
		\bottomrule
	\end{tabularx}
\end{table}

Effective brokerage strategies may require complex combinations and sequences of different brokerage behaviours over time. Skilled brokerage often involves selective deployment of these approaches with different actors or for different objectives. Different combinations of \emph{tertius iungens} and \emph{tertius gaudens} behaviour are necessary to tailor brokerage strategies to match the situation \citep{obstfeld2014brokerage}. 



%lingo2010

\section{Summary}

Relative differences in absorptive capacity can impede the flow of knowledge across organisational boundaries, create power imbalances, and undermine alliance performance. Tacit knowledge is key in overcoming relative differences in absorptive capacity in open innovation partnerships. Communicating tacit knowledge requires significant effort and people need to be sufficiently motivated to share or seek out tacit knowledge. Tacit knowledge is a source of power. Because it is a valuable resource accumulated over a lifetime of experience, people are not going to hand over their tacit knowledge just like that. They will be very selective in who they choose to share their tacit knowledge with. People who share their tacit knowledge are empowering others. They are more likely to share their tacit knowledge with others they deem trustworthy. Tacit knowledge is more likely to be shared between actors whose power is not too imbalanced.

% reciprocity
% swift trust
% brokerage

