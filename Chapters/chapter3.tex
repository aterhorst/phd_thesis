% Chapter 3 - Network perspective

\section{Introduction}

The previous chapter described how absorptive capacity has evolved over time. It also explained how the advent of open innovation helped ground the concept and highlighted the critical role of tacit knowledge in building absorptive capacity. Since a firm's absorptive capacity is partly determined by its ability to establish and maintain productive inter-organisational networks \citep{inkpen2005social}, the previous chapter with a suggestion to use social network analysis to assess practices that build absorptive capacity in open innovation. \medskip

\section{Social networks}

Social networks provide a way of thinking about social systems, one that focuses on the relationships among entities that make up the system \citep{borgatti2013analyzing,robins2015doing}. Such networks can be represented mathematically as graphs consisting of a set of vertices and a set of edges that connect vertices \citep{newman2010networks}. \medskip

Vertices represent actors or nodes in a social network, which can be individuals, groups, organisations, regions, or even nations. Actors may be distinguished by binary, categorical or continuous attributes. For example, consider an individual actor classified as female (binary attribute), who works for a particular organisation (categorical attribute), with a specific number of years work experience (continuous attribute). \medskip

Edges represent relations or social ties between actors. Ties can be measured as directed or undirected and as binary or valued. Deciding whether to measure a tie as directed or undirected depends on the theoretical nature of the tie. For instance, co-membership is inherently undirected whereas authority is essentially directed. Directed and undirected ties can be measured as binary ties that either exist or do not exist, or as valued ties that can be stronger or weaker, transmit more or fewer resources, or have greater or lesser amount of contact \citep{scott2011sage}.\medskip

Different types of relations may exist between actors with each type of relation giving rise to a corresponding network \citep{borgatti2013analyzing}. Measuring knowledge sharing ties would, for example, generate a knowledge sharing network. Assigning an attribute to the knowledge sharing tie allows us to qualify the relationship in terms of, say, the content or frequency of knowledge sharing (e.g. how much of the knowledge be shared is tacit in nature). Ties can be classified according to similarities between actors or by relational roles, relational cognition, and events \citep{borgatti2013analyzing}. \medskip

Ties between actors who share something in common (e.g. work at the same location, are affiliated to the same body, participate in the same event, or share a common attribute) are referred to as \enquote{similarity ties}. Relational roles include kinship and other ties, such as friendship, advice, and managerial ties. Relational cognition refers to ties that are affective (e.g. like or dislike another actor) or perceptual (e.g. belief about the other actor) in nature. Relational events refer to ties defined by specific social interactions (e.g. a transaction of some kind) and flows (e.g. knowledge flows). Actors who know each other well are said to have strong ties with one another. Ties that are characterised by infrequent interaction, short histories, and limited emotional closeness may be characterised as weak ties \citep{baer2010strength}. 

Some ties are dependent on others. An example is friendship, which usually develops because of an pre-existing similarity tie (e.g. both actors live in the same neighbourhood, attend the same school, or work at the same place) or via a relational event tie (e.g. actors were introduced to each other at a specific event or worked together on a particular project). Actors are more likely to share knowledge (relational event) with others who have common interests (similarity tie) or with others they trust (relational cognition). \medskip 

\section{Knowledge networks} % ahuja, phelps, krackhardt (strong ties), argote

Actors in knowledge networks serve both as keepers of knowledge and as agents that seek out, communicate, and create knowledge \citep{phelps2012knowledge,pugh2013designing}. Though a well-educated actor with many years relevant work experience is expected to possess much knowledge, this knowledge only becomes valuable when it is used or enacted in practice \citep{marabelli2012knowledge,freeman2015knowledge}. 

Examining the local configuration of knowledge networks can shed light on the social processes that transform new knowledge into innovations. For instance, actors that know each other well are said to have strong ties with one another. Such actors tend to have similar interests and are privy to the same knowledge. Strong ties tend to make people look inward and not be very receptive to external knowledge. Casual acquaintances, on the other hand, can be regarded as weak ties. Because acquaintances usually mix in different social circles, weak ties are more likely to provide actors access to new knowledge and opportunities \citep{granovetter1973strength}. \medskip

Knowledge networks with an abundance of weak ties are typically sparse with many disconnected parts or \enquote{structural holes} \citep{burt1992structural}. Actors who bridge otherwise disconnected parts of the knowledge network are termed knowledge brokers. Brokerage may be defined as the \enquote{behaviour by which an actor influences, manages, or facilitates interactions between other actors} \citep{obstfeld2014brokerage}. Knowledge brokers make connections between those who need knowledge and those who have it \citep{davenport1998successful}. Not only do they connect otherwise disconnected groups of people, they also facilitate the translation, integration and combination of diverse knowledge \citep{davis2010agency}. \medskip

Knowledge brokers occupy favourable positions in knowledge networks. They have access to new and diverse knowledge

Some knowledge brokers exploit this to their own advantage while others 

Though brokerage across structural holes in a disconnected network is a source of value, networks with fewer structural holes engender trust and reduce opportunism, leading to more productive collaborations \citep{ahuja2000collaboration}. Strong ties are more effective than weak ties in enhancing knowledge transfer and learning as well as an individual’s ability to benefit from collaborating with diverse partners \citep{rost2011strength,phelps2012knowledge,tortoriello2012bridging}. Clarifying the implications of closed versus disconnected network structures for various organisational outcomes is important to our understanding of network resources \citep{ahuja2000collaboration}. \medskip

\section{Social network analysis}

The aim of social network analysis is  \citep{de2011exploratory}. Most basic questions in social network analysis involve the measurement and modelling of particular structural properties \citep{butts2008social}. To fully understand the implications of social relations between actors, one has to consider how these are shaped by broader patterns of social interaction \citep{scott2011sage}. This requires more than simply measuring the basic characteristics of networks. A set of assumptions is needed to best describe and explain social phenomena of interest. One may apply an existing socio-psychological theory to test hypotheses about social relations. Alternatively, one may use networks to explain specific outcomes or assess how social processes are influenced by network effects \citep{scott2011sage,borgatti2013analyzing}. \medskip

\section{Exponential random graph models}

Exponential random graph models (ERGMs) are a class of statistical model for social networks originally developed by \citet{frank1986markov} and refined by \citet{wasserman1996logit} and \citet{pattison1999logit}. ERGMs have the capacity to address complex social structures. Recent model derivations are able to examine both individual-level variables and structural relations simultaneously \citep{robins2007recent}. ERGMs are state-of-the-art as far as social network analysis is concerned \citep{lusher2013exponential}. \medskip

An ERGM is essentially a pattern recognition device which breaks a network down into its constituent network motifs or configurations and then test if particular configurations occur more or less frequently than would be expected by chance alone. Network configurations are patterns of social network ties assumed to represent underlying social processes or mechanisms \citep{lusher2014cooperative}. For example, a theory may suggest that actors with specific attributes are more likely to receive social ties. This can be tested using an ERGM to see if actors with such attributes are receiving more ties than would be expected by chance alone. That way, a researcher can test certain hypotheses or propositions about tie formation relating to theory \citep{robins2007recent}. \medskip

Importantly, ERGMs permit differentiation between structural network effects and processes related to actor attributes. This is because ERGMs are underpinned by the assumption of conditional dependence, as opposed to the assumption of independence of observations in standard statistical tests \citep{pattison2002neighborhood}. An ERGM is similar to a logistic regression, but is more sophisticated because it can handle complex dependency assumptions. This reflects social reality where ties are largely inter-dependent (e.g. friendship, which as stated earlier, usually emerges as a consequence of a pre-existing similarity tie). \medskip

Purely structural effects reflect self-organising or endogenous processes in which ties form due to the presence or absence of other ties. In the case of reciprocity, for example, because one person has first done someone a favour, that other person is more likely to reciprocate the favour (\enquote{you scratch my back because I have scratched yours}). One tie follows on from the other i.e. one tie is dependent upon the other. Ties may also form due to actor attributes and are known as actor-relation effects in ERGMs. The tendency of individuals to associate and bond with similar others (also known as \enquote{homophily}) is an example of an actor-relation effect that can be used to explain why actors in an open innovation project might have a propensity to share knowledge with others from the same organisation as themselves. 

ERGMs provide a more principled way of making inferences about the association between actor attributes and network ties because ERGMs can distinguish between ties formed due to actor attributes or whether an actor’s popularity is the result of being embedded within many purely structural network structures \citep{lusher2013exponential}. Other methods used to assess the effect of actor attributes on network structures, such as linear regression, are unable to make such distinctions, and are thus more limited regarding the conclusions such methods can draw. A further advantage of ERGMs is that multiple explanations for network tie formation can be examined simultaneously in one model, comparing one effect against another to see which is more likely to be associated with the formation of network ties (e.g. is it age or experience that matters in advice-seeking?).