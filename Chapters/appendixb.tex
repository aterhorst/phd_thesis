\begin{landscape}
	\renewcommand\thetable{A\arabic{table}} 
	\begin{small}

\begin{center}
	\begin{longtable}{c P{9cm} P{9cm} P{3cm}}
% 	\begin{longtable}{c p{9cm} p{9cm} p{3cm}}
		\caption{Semi-structured interview questions.} \label{tab:interview} \\
		\toprule \multicolumn{1}{c}{Question No.} & \multicolumn{1}{c}{Main Question} & \multicolumn{1}{c}{Probe(s)} & \multicolumn{1}{c}{Construct} \\ \midrule
		\endfirsthead
		
		\multicolumn{4}{c}%
		{{ \tablename\ \thetable{} -- continued from previous page}} \\ \\
		\toprule \multicolumn{1}{c}{Question No.} & \multicolumn{1}{c}{Question} & \multicolumn{1}{c}{Probe(s)} & \multicolumn{1}{c}{Construct} \\
		\midrule
		\endhead
		\midrule
		\multicolumn{4}{r}{{Continued...}} \\ 
		\endfoot
		\bottomrule
		\endlastfoot

1 & Thank you for making time to be interviewed. I am researching collaborative or open innovation. This is the sharing of knowledge and ideas between different organisations to develop new products or services that solve a problem. The aim of this interview is to explore some of the factors that influence sharing knowledge in your current collaboration. By shedding light on these factors, my research will hopefully lead to more effective processes for managing the flow of knowledge in open innovation. Do you have any questions? & & About the research \\
2 & Before we begin, I need your consent to use the information from this interview in my study. I will record our discussion and have it transcribed so that I can use it in my analysis. Nobody aside from myself, the transcription service, or my supervisor/s will have access to the audio or transcript. Your participation is voluntary and you can choose to stop the interview at any stage. Do you have any questions? Please sign the consent form if you are happy to proceed. & & Consent \\
& \textit{Switch audio recorder on.} & & \\
3 & Please tell me how this collaboration came about. & & Chronology of events. \\
4 & How do you think the partners benefit from this collaboration? & What is the value proposition for each partner? Will they benefit materially? What about being exposed to new ways of doing things? & Value proposition. \\
5 & How would you describe the progress this collaboration has made to date? & Describe key events and milestones. & Chronology of events.\\
6 & Please tell me what you do in this collaboration. & Provide an example of the work you do in this collaboration. How will you contribute to the success of this collaboration? & Self-efficacy. \\
7 & How did you become involved in this collaboration? & Why did you become involved? How do you benefit personally from being involved? Have any of these benefits materialised?  Give examples. & Motivation. \\
8 & What do you enjoy the most about this collaboration? & & Motivation. \\
9 & What does innovation mean to you? & Provide an example of an innovation. How conducive is the collaboration for innovation? & Shared understanding. \\
10 & How are the different agendas of collaboration partners recognised and managed? & How are these different perspectives reconciled?  How inclusive is decision-making? Is there a dominant partner? Give examples. & Power. \\
11 & Please tell me how important your relationship with [nominated alter] is and why? & & Tie strength \\
12 & Please explain how you share information and/or know-how with [nominated alter]. & Describe the type of information and/or knowledge being shared. How much of the knowledge is documented? How is know-how shared? & Knowledge acquisition/assimilation. \\
13 & Explain how information and/or know-how provided by [nominated alter] helps you accomplish your tasks in this collaboration? Give an example. & & \\
14 & How does trust influence your decision to share information and/or know-how with [nominated alter]. & Do you share knowledge because you are friends with [nominated alter] or because you think [nominated alter] will use this knowledge to good effect? & Affect- and cognition-based trust. \\
15 & How does trust influence your decision to accept information and/or know-how provided by [nominated alter]. & Do you accept knowledge because [nominated alter] is a friend or because [nominated alter] is a good source of knowledge? & Affect- and cognition-based trust. \\
16 & How do you establish trust in short-term collaborations? & & Affect- and cognition-based trust. \\
17 & How do you create new ideas with [nominated alter]? & Explain the process. Give an example. How do you reconcile different approaches to problem solving? & Knowledge assimilation/transformation. \\
18 & How do you implement ideas with [nominated alter]? & Explain the process. Give an example. & Knowledge transformation/exploitation. \\
19 & How do you see knowledge sharing in this collaboration? & What does this reveal about learning processes in this collaboration? Has this influenced how you perceive learning in a dynamic stakeholder environment? & Collaboration strength. \\
20 & How open are people in this collaboration to new external knowledge provided by partner and other organisations? & & Not-invented-here syndrome. \\
21 & How willing are people in this collaboration to pass on their knowledge and/or know-how to people in other organisations? & & Not-sold-here syndrome. \\
22 & How complex is the knowledge required for this collaboration? & How difficult is it to transfer knowledge and/or related know-how? How easily is knowledge and/or related know-how absorbed by others? & Tacit knowledge. \\
23 & Please tell me how organisational learning processes differ between the different collaboration partners. & & Organisational learning. \\
24 & How do you see knowledge being transformed into commercial outcomes in this collaboration? & How do preconceived ideas and/or biases affect outcomes? Give an example & Knowledge transformation/exploitation. \\
25 & What is your perspective on the strength of this collaboration? & Coordination of activities? Transmission of new knowledge? Co-creation of new ideas? How can things be improved? Is there opportunity for other collaborations? & Collaboration strength. \\
26 & What do you think are the boundaries that impact this collaboration? & How tangible are these boundaries? Provide examples where boundaries may inhibit the flow of information, knowledge and/or know-how. & Boundary-spanning. \\
27 & How do boundaries affect interactions between others in this collaboration? & How do people facilitate the transfer of knowledge, generation of ideas across boundaries? Give examples. & Boundary spanning. \\
28 & How do you see your organisation integrating collaborative innovation into its business model in the future? & & Business model innovation. \\
29 & Do you have anything more to say about knowledge sharing in this collaboration? & & Knowledge acquisition/assimilation. \\
30 & Do you have anything more to say about organisational learning in this collaboration? & & Organisational learning. \\
31 & Do you have anything more to say about the innovation culture of this collaboration? & & Innovation culture. \\
32 & I have no further questions. Is there anything else you would like to bring up, or ask about, before we finish this interview? & & Closure. \\
& \textit{Switch audio recorder off.} & & \\
33 & Thank you for your time and for participating in this study. I may contact you later to clarify some things you said during the interview. I will email you a copy of the interview transcripts for your records. & & Closure. \\.
\end{longtable}
\end{center}
\end{small}
\end{landscape}
