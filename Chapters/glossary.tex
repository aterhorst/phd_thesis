

\begin{longtable}{l p{10cm}}
	\textbf{Absorptive capacity} & The ability of a firm to acquire, assimilate, and exploit new external knowledge for commercial ends. May be viewed as a specific example of organisational learning concerning new external knowledge.\\ 
	\textbf{Ethnography} & \\
	\textbf{Open innovation} & An emerging paradigm that assumes organisations can and should use external ideas as well as internal ideas, and internal and external paths to market, to advance their technology. More formally defined as a distributed innovation process based on purposively managed knowledge flows across organisational boundaries, using pecuniary and non-pecuniary mechanisms in line with the organisation’s business model. \\ 
	\textbf{Organisational learning} & \\
	\textbf{Self-determination theory} & \\
	\textbf{Social network} & A social structure made of actors that are generally individuals or organisations. A social network represents relationships and flows between people, groups, organisations, animals, computers or other information/knowledge processing entities.\\
	\textbf{Social network analysis} &  Maps and measures relationships and flows between actors. Social network analysis provides both a visual and a mathematical analysis of actor relationships \\
	\textbf{Sticky knowledge} & \\
	\textbf{Tacit knowledge} & \\
	\textbf{Theory of planned behaviour} & \\
	
\end{longtable} 