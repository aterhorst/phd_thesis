% Chapter 2 - Social network perspective
\section{Introduction}

Transforming knowledge into valuable innovations depends very much on the configuration of knowledge networks \citep{tortoriello2010activating}. We can use social network analysis to examine network configurations and determine the extent to which these are shaped by individual attributes and knowledge properties. This chapter describes some of the salient features of social networks and key social mechanisms for mobilising knowledge. It then introduces exponential random graph models, an advanced social network analysis technique for examining network configurations. \medskip  

\section{Social networks}

Social networks provide a way of thinking about social systems, one that focuses on the relationships among entities that make up the system \citep{borgatti2013analyzing,robins2015doing}. Such networks can be represented mathematically as graphs consisting of a set of vertices and a set of edges that connect vertices \citep{newman2010networks}. \medskip

Vertices represent actors or nodes in a social network, which can be individuals, groups, organisations, regions, or even nations. Actors may be distinguished by binary, categorical or continuous attributes. For example, consider an individual actor classified as female (binary attribute), who works for a particular organisation (categorical attribute), with a specific number of years work experience (continuous attribute). \medskip

Edges represent relations or social ties between actors. Ties can be measured as directed or undirected and as binary or valued. Deciding whether to measure a tie as directed or undirected depends on the theoretical nature of the tie. For instance, co-membership is inherently undirected whereas authority is essentially directed. Directed and undirected ties can be measured as binary ties that either exist or do not exist, or as valued ties that can be stronger or weaker, transmit more or fewer resources, or have greater or lesser amount of contact \citep{scott2011sage}.\medskip

Different types of relations may exist between actors with each type of relation giving rise to a corresponding network \citep{borgatti2013analyzing}. Measuring knowledge sharing ties would, for example, generate a knowledge sharing network. Assigning an attribute to the knowledge sharing tie allows us to qualify the relationship in terms of, say, the content or frequency of knowledge sharing (e.g. how much of the knowledge be shared is tacit in nature). \medskip

Ties can be continuous or  according to similarities between actors or by relational roles, relational cognition, and events \citep{borgatti2013analyzing}. \medskip

Ties between actors who share something in common (e.g. work at the same location, are affiliated to the same body, participate in the same event, or share a common attribute) are referred to as \enquote{similarity ties}. Relational roles include kinship and other ties, such as friendship, advice, and managerial ties. Relational cognition refers to ties that are affective (e.g. like or dislike another actor) or perceptual (e.g. belief about the other actor) in nature. Relational events refer to ties defined by specific social interactions (e.g. a transaction of some kind) and flows (e.g. knowledge flows). Actors who know each other well are said to have strong ties with one another. Ties that are characterised by infrequent interaction, short histories, and limited emotional closeness may be characterised as weak ties \citep{baer2010strength}. 

\begin{figure}
	\centering
	\includegraphics[width=0.9\linewidth]{tie_type}
	\caption{Different types of ties. Modified after \citet{borgatti2009network}.}
	\label{fig:tie_type}
\end{figure}


Some ties are dependent on others. An example is friendship, which usually develops because of an pre-existing similarity tie (e.g. both actors live in the same neighbourhood, attend the same school, or work at the same place) or via a relational event tie (e.g. actors were introduced to each other at a specific event or worked together on a particular project). Actors are more likely to share knowledge (relational event) with others who have common interests (similarity tie) or with others they trust (relational cognition). \medskip 

\section{Knowledge networks}

Actors in knowledge networks serve both as keepers of knowledge and as agents that seek out, communicate, and create knowledge \citep{phelps2012knowledge,pugh2013designing}. Knowledge by itself is simply a latent resource. It only becomes valuable once it is mobilised and used \citep{marabelli2012knowledge,freeman2015knowledge}. 

Examining the local configuration of knowledge networks can shed light on the social processes that transform new knowledge into innovations. For instance, actors that know each other well are said to have strong ties with one another. Such actors tend to have similar interests and are privy to the same knowledge. Strong ties tend to make people look inward and not be very receptive to external knowledge. Casual acquaintances, on the other hand, can be regarded as weak ties. Because acquaintances usually mix in different social circles, weak ties are more likely to provide actors access to new knowledge and opportunities \citep{granovetter1973strength}. \medskip

Knowledge networks with an abundance of weak ties are typically sparse with many disconnected parts or \enquote{structural holes} \citep{burt1992structural}. Actors who bridge otherwise disconnected parts of the knowledge network are termed knowledge brokers. They make connections between those who need knowledge and those who have it \citep{davenport1998successful}. Not only do knowledge brokers connect otherwise disconnected groups of people, they also facilitate the translation, integration and combination of diverse knowledge \citep{davis2010agency}.
Knowledge brokers are able to identify and establish strategic relationships with keepers of knowledge. Some knowledge brokers exploit this to their own advantage while others try establish new relations between otherwise disconnected people \citep{gould1989structures,burt1992structural,obstfeld2014brokerage}. \medskip

Though brokerage across structural holes in a disconnected network is a source of value, networks with fewer structural holes engender trust and reduce opportunism, leading to more productive collaborations \citep{ahuja2000collaboration}. Strong ties are more effective than weak ties in enhancing knowledge transfer and learning as well as an individual’s ability to benefit from collaborating with diverse partners \citep{rost2011strength,phelps2012knowledge,tortoriello2012bridging}. \medskip

\section{Social network analysis}

The aim of social network analysis is detecting and interpreting patterns of social ties between actors \citep{wasserman1994social,de2011exploratory}. Most basic questions in social network analysis involve the measurement and modelling of particular structural properties such as reciprocity, degree distribution, cyclic and transitive closure, and homophily \citep{butts2008social,snijders2011statistical}.

Reciprocity is a basic feature of social networks. Social exchange theory \citep{emerson1976social} and game theory (Axelrod, 1984) suggest dyadic relationships have a tendency to be symmetrical. Reciprocity is a way of achieving balanced and more stable relations. Non-reciprocated or asymmetric ties are more likely to be unstable.

Degree distribution considers actors who are either highly connected and or have few connections. and this led to models for node centrality reviewed by Freeman (1979). An important theoretical account was the rich-get-richer phenomenon (de Solla Price 1976). This will lead to a high dispersion of the nodal degrees, which then may further lead to core-periphery structures (Borgatti and Everett, 1999) or various other types of hierarchical structures. 

Transitivity of ties is expressed by the saying \enquote{friends of my friends are my friends}, and was proposed as an essential element of networks by Rapoport (1953a,b). Davis (1970) found large-scale empirical support for transitivity in networks. 

Transitivity is rooted deeply in sociology, going back to authors such as Simmel (1950) and elaborated more recently by Coleman (1990). Transitivity therefore also has been called clustering (Watts, 1999). Transitivity can have two faces: it may point to a hierarchical ordering or to a clustered structure. These two can be differentiated by the aid of the number of 3-cycles. A relatively high number of 3-cycles points toward clustering, a relatively low number toward hierarchy. Davis (1970) found empirically that social networks tend to contain a relatively low number of 3-cycles, indicating the pervasiveness of hierarchies in social networks. Hierarchies in directed networks, as exhibited by high transitivity and few 3-cycles, may be local or global. A global hierarchy will be indicated by the ordering of the in-degrees and/or out-degrees, where the typical pattern e.g., in esteem or advice asking, is directed from low to high. In a purely global hierarchy, which can be seen, e.g., in some advice networks, in a statistical model the degree differentials will be sufficient to explain the low number of 3-cycles. But local hierarchies are possible in directed networks even when the in-degrees and out-degrees exhibit rather little variability.

Homophily is the tendency of similar actors to relate to each other \citep{mcpherson2001birds}. Theoretical arguments can be based on opportunity, affinity, ease of communication, reduced transaction costs and break-off risks, and organisational foci composed of similar individuals. This leads to a higher probability of ties being formed between actors with similar values on relevant covariates \citep{snijders2011statistical}.

To fully understand the implications of social relations between actors, one has to consider how these are shaped by broader patterns of social interaction \citep{scott2011sage}. This requires more than simply measuring the basic characteristics of networks. A set of assumptions is needed to best describe and explain social phenomena of interest. This may involve applying an existing psychological or social theory to test hypotheses about social relations. Alternatively, one may use networks to explain specific outcomes or assess how social processes are influenced by network effects \citep{scott2011sage,borgatti2013analyzing}. \medskip

\section{Exponential random graph models}

Exponential random graph models (ERGMs) are a class of statistical model for social networks originally developed by \citet{frank1986markov} and refined by \citet{wasserman1996logit} and \citet{pattison1999logit}. ERGMs have the capacity to address complex social structures. Recent model derivations are able to examine both individual-level variables and structural relations simultaneously \citep{robins2007recent}. \medskip

An ERGM is essentially a pattern recognition device which breaks a network down into its constituent network motifs or configurations and then test if particular configurations occur more or less frequently than would be expected by chance alone. Network configurations are patterns of social network ties assumed to represent underlying social processes or mechanisms \citep{lusher2014cooperative}. For example, a theory may suggest that actors with specific attributes are more likely to receive social ties. This can be tested using an ERGM to see if actors with such attributes are receiving more ties than would be expected by chance alone. That way, a researcher can test certain hypotheses or propositions about tie formation relating to theory \citep{robins2007recent}. \medskip

ERGMs permit differentiation between structural network effects and processes related to actor attributes. Unlike the assumption of independence of observations in standard statistical tests, ERGMs are based on the assumption of conditional dependence \citep{pattison2002neighborhood}. An ERGM is similar to a logistic regression, but is more sophisticated because it can handle complex dependency assumptions. This reflects social reality where ties are largely interdependent \citep{kadushin2012understanding}. \medskip

Purely structural effects reflect self-organising or endogenous processes in which ties form due to the presence or absence of other ties. In the case of reciprocity, for example, because one person has first done someone a favour, that other person is more likely to reciprocate the favour (\enquote{you scratch my back because I have scratched yours}). One tie follows on from the other i.e. one tie is dependent upon the other. Ties may also form due to actor attributes and are known as actor-relation effects in ERGMs. The tendency of individuals to associate and bond with similar others (also known as \enquote{homophily}) is an example of an actor-relation effect. \medskip 

\begin{table}[]
	\tiny
	\centering
	\caption{Exponential random graph model parameters used in this study.}
	\label{erm_params}
	\begin{tabular}{lcl}
		\toprule
		Parameter & Graphic & Explanation  \\ \midrule
		\textbf{Purely structural effects} & & \\
		Arc (edge)                    	& \begin{minipage}{.2\textwidth} \centering \includegraphics[width=0.4\linewidth]{Images/Arc} \end{minipage} 					& \begin{tabular}[c]{l}Baseline propensity for a tie to form in the absence of other\\ effects.\end{tabular} \\ \\
		Reciprocity (mutuality)       	& \begin{minipage}{.2\textwidth} \centering \includegraphics[width=0.4\linewidth]{Images/Reciprocity} \end{minipage} 			& \begin{tabular}[c]{l}Propensity for a tie from one actor to a second when there\\ is already a tie from the second to the first.\end{tabular} \\ \\                                                                                                       \\
		TwoPath (simple connectivity) 	& \begin{minipage}{.2\textwidth} \centering \includegraphics[width=0.4\linewidth]{Images/Twopath} \end{minipage}        		& \begin{tabular}[c]{l}Propensity for ties to form as part of simple path formations. \end{tabular} \\ \\
		AinS (popularity spread)      	& \begin{minipage}{.2\textwidth} \centering \includegraphics[width=0.6\linewidth]{Images/AinS} \end{minipage}        			& \begin{tabular}[c]{l}Propensity for dispersion in the in-degree distribution,\\ indicating there are a few highly popular actors. \end{tabular} \\ \\
		AoutS (activity spread)       	& \begin{minipage}{.2\textwidth} \centering \includegraphics[width=0.6\linewidth]{Images/AoutS} \end{minipage}  				& \begin{tabular}[c]{l}Propensity for dispersion in the out-degree distribution,\\ indicating there are a few highly active actors. \end{tabular}  \\ \\
		AT-T (path closure)           	& \begin{minipage}{.2\textwidth} \centering \includegraphics[width=0.6\linewidth]{Images/AT-T} \end{minipage}        			& \begin{tabular}[c]{l}Propensity for ties to form as part of transitive triad or a\\ multiple transitive configuration. \end{tabular} \\ \\
		AT-C (cyclic closure)         	& \begin{minipage}{.2\textwidth} \centering \includegraphics[width=0.6\linewidth]{Images/AT-C} \end{minipage}        			& \begin{tabular}[c]{l}Propensity for ties to form as part of a cyclic triad or a\\ multiple cyclic configuration. \end{tabular} \\ \\
		A2P (multiple connectivity)   	& \begin{minipage}{.2\textwidth} \centering \includegraphics[width=0.6\linewidth]{Images/A2P} \end{minipage}       				& \begin{tabular}[c]{l}Propensity for ties to form as part of formations involving\\ multiple short paths between actors. \end{tabular} \\ \\
		\textbf{Actor-relation effects} & & \\
		Attribute sender              	& \begin{minipage}{.2\textwidth} \centering \includegraphics[width=0.4\linewidth]{Images/Sender} \end{minipage}        			& \begin{tabular}[c]{l}Propensity for a tie to be directed from an actor with a\\ particular attribute. \end{tabular} \\ \\
		Attribute receiver             	& \begin{minipage}{.2\textwidth} \centering \includegraphics[width=0.4\linewidth]{Images/Receiver} \end{minipage}        		& \begin{tabular}[c]{l}Propensity for a tie to be directed toward an actor with a\\ particular attribute. \end{tabular} \\ \\
		Attribute match               	& \begin{minipage}{.2\textwidth} \centering \includegraphics[width=0.4\linewidth]{Images/Match} \end{minipage}       			& \begin{tabular}[c]{l}Propensity for a tie to form between actors with the same\\ categorical attribute.\end{tabular} \\ \\
		Attribute mismatch reciprocity 	& \begin{minipage}{.2\textwidth} \centering \includegraphics[width=0.4\linewidth]{Images/MisMatchReciprocity} \end{minipage}  	& \begin{tabular}[c]{l}Propensity for a tie to form between actors with a\\ non-matching categorical attribute.\end{tabular} \\ \\
		Attribute difference          	& \begin{minipage}{.2\textwidth} \centering \includegraphics[width=0.4\linewidth]{Images/Difference} \end{minipage}       		& \begin{tabular}[c]{l}Propensity for a tie to form between actors with a similar\\ continuous attribute. \end{tabular} \\ \\
		Attribute product             	& \begin{minipage}{.2\textwidth} \centering \includegraphics[width=0.4\linewidth]{Images/Product} \end{minipage}        		& \begin{tabular}[c]{l}Propensity for a tie to form between actors who both score\\ highly on the same continuous attribute. \end{tabular} \\ \\
		\textbf{Actor-brokerage effects} & & \\
		b\textsubscript{O} (liaison role)			      	&  \begin{minipage}{.2\textwidth} \centering \includegraphics[width=0.4\linewidth]{Images/b_O} \end{minipage}   & \begin{tabular}[c]{l}Propensity to have brokers who mediate communication\\ between two individuals from different groups, neither of\\ which they belong to.\end{tabular}\\ \\
		b\textsubscript{IO} (representative role)		   	& \begin{minipage}{.2\textwidth} \centering \includegraphics[width=0.4\linewidth]{Images/b_IO} \end{minipage}   & \begin{tabular}[c]{l}Propensity to have brokers who mediate communication\\ from in-group members to out-group members.\end{tabular}\\ \\
		b\textsubscript{OI} (gatekeeper role) 				& \begin{minipage}{.2\textwidth} \centering \includegraphics[width=0.4\linewidth]{Images/b_OI} \end{minipage}   & \begin{tabular}[c]{l}Propensity to have brokers who mediate communication\\ from out-group members to in-group members. \end{tabular}\\ \\
		w{\textsubscript{O} (itinerant broker)			&  \begin{minipage}{.2\textwidth} \centering \includegraphics[width=0.4\linewidth]{Images/w_O} \end{minipage}   & \begin{tabular}[c]{l}Propensity to have brokers who mediate communication\\ between two individuals from a single group to which they\\ do not belong. \end{tabular}\\ 
		w\textsubscript{I} (coordination role)				& \begin{minipage}{.2\textwidth} \centering \includegraphics[width=0.4\linewidth]{Images/w_I} \end{minipage}    & \begin{tabular}[c]{l}Propensity to have brokers who mediate communication\\ between two individuals from his or her own group. \end{tabular}\\ \\	
		\textbf{Network covariate effects} & & \\
		Dyadic covariate             	& \begin{minipage}{.2\textwidth} \centering \includegraphics[width=0.4\linewidth]{Images/DyadicCovariate} \end{minipage}    	& \begin{tabular}[c]{l}Propensity for a tie of one type to form from one actor to\\ another if a tie of another type is already present, though\\ the covariate network is fixed (i.e. exogenous) in the\\ model, and so cannot vary. \end{tabular} \\\bottomrule                                                                                                       
	\end{tabular}
\end{table}

ERGMs provide a more principled way of making inferences about the association between actor attributes and network ties because ERGMs can distinguish between ties formed due to actor attributes or whether an actor’s popularity is the result of being embedded within other purely structural network structures \citep{lusher2013exponential}. Alternative methods used to assess the effect of actor attributes on network structures, such as linear regression, are unable to make such distinctions, and are thus more limited regarding the conclusions such methods can draw. ERGMs also allow multiple explanations for network tie formation to be examined simultaneously in one model, comparing one effect against another to see which is more likely to be associated with the formation of network ties (e.g. is it age or experience that matters in advice-seeking?). \medskip

\section{Conclusion}

The micro-structure of knowledge-sharing networks should reveal much regarding the social mechanisms of knowledge acquisition, assimilation, transformation, and application \citep{reagans2003network,phelps2012knowledge,tortoriello2010activating,tortoriello2015social}. For example, a researcher could examine how individual attributes and knowledge properties relate to specific patterns of social interaction. The influence of power-relations on absorptive capacity can be assessed by examining patterns of knowledge brokerage \citep{burt2004structural,obstfeld2005social,obstfeld2014brokerage}. Examining closure in tacit knowledge networks can help explain knowledge assimilation processes and the level of collaboration. Chapter 3  introduces social network analysis and explains how this can be used to assess practices that build absorptive capacity in open-innovation collaborations. 

Clarifying the implications of closed versus disconnected network structures for various organisational outcomes is important to our understanding of network resources \citep{ahuja2000collaboration}. \medskip