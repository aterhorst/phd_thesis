% Discussion


The Dunning–Kruger effect is a cognitive bias in which low-ability individuals suffer from illusory superiority, mistakenly assessing their ability as much higher than it really is. Dunning and Kruger attributed this bias to a metacognitive inability of those of low ability to recognize their ineptitude and evaluate their ability accurately. Their research also suggests corollaries: high-ability individuals may underestimate their relative competence and may erroneously assume that tasks which are easy for them are also easy for others.