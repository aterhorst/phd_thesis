\section{Introduction}

Chapter 5 described each of the three cases and contrasted these in terms of their maturity, type of open innovation, demographics, and knowledge provider ties. It revealed the three cases are quite different in terms of the type of open innovation being carried out, how far each had travelled on their open innovation journey, the geographic spread of participants, and the nature of tacit and explicit knowledge exchanges. \medskip

This chapter presents the results of the exponential random graph modelling. The models aim to show which network configurations explain the global structure of observed networks. Recall that network configurations represent underlying social processes or mechanisms. The propositions presented in Chapters 2 and 3 refer to specific social processes or mechanisms pertinent to tacit knowledge sharing in open innovation. Two sets of modelling were done. One set tested propositions about the role of motivation, trust, and power in tacit and explicit knowledge exchanges, while the other set examined the significance of different broker roles in each case. Readers need to be reminded that the online survey asked respondents to name others who provided them with useful and relevant knowledge. As explained in section on data pre-processing in Chapter 4 (Section 4.5.2.5), ties were reversed to depict named knowledge providers as senders of knowledge. \medskip

From a critical realist perspective, the exponential random graph modelling addresses the observed or experienced reality i.e. knowledge sharing events in each case. \medskip

\begin{sidewaystable}[p]
\centering
\resizebox{0.9\textwidth}{!}{%	
\begin{threeparttable}
\footnotesize
\setlength{\tabcolsep}{6pt}
\renewcommand{\arraystretch}{1}
\caption[Parameter estimates for the first set of ERGMs]{ERGM parameter estimates for models exploring motivation, trust and power in tacit and explicit knowledge provider networks. Refer to Table \ref{tab:ergm_params} for an explanation of network parameters.}
\label{tab:ergm_1}
\begin{tabular}{@{}lrrcrrcrr@{}}
\toprule
& \multicolumn{2}{c}{Case 1} &  & \multicolumn{2}{c}{Case 2} &  & \multicolumn{2}{c}{Case 3} \\ \cmidrule(lr){2-3} \cmidrule(lr){5-6} \cmidrule(l){8-9} \multicolumn{1}{c}{} & \multicolumn{1}{c}{Tacit} & \multicolumn{1}{c}{Explicit} & \multicolumn{1}{c}{} & \multicolumn{1}{c}{Tacit} & \multicolumn{1}{c}	{Explicit} & \multicolumn{1}{c}{} & \multicolumn{1}{c}{Tacit} & \multicolumn{1}{c}{Explicit} \\
\cmidrule(lr){2-3} \cmidrule(lr){5-6} \cmidrule(l){8-9} 
\textbf{Purely structural effects (endogenous)} & \multicolumn{1}{l}{} & \multicolumn{1}{l}{} &  & \multicolumn{1}{l}{} & \multicolumn{1}{l}{} &  & \multicolumn{1}{l}{} & \multicolumn{1}{l}{} \\
Arc (edge) & -16.13 (4.39)* & 9.41 (7.03)\phantom{*} &  & -4.59 (1.4)* & -3.73 (1.94)\phantom{*} &  & -7.33 (1.47)* & -2.41 (1.21)\phantom{*} \\
Reciprocity (mutuality) & -1.76 (1.29)\phantom{*} & -2.04 (1.19)\phantom{*} &  & -0.3 (0.65)\phantom{*} & -0.16 (0.93)\phantom{*} &  & -0.45 (0.62)\phantom{*} & -0.75 (0.46)\phantom{*} \\
TwoPath (simple connectivity) & -2.63 (1.60)\phantom{*} & -0.25 (0.24)\phantom{*} &  & -0.01 (0.03)\phantom{*} & 0.08 (0.07)\phantom{*} &  & -0.24 (0.13)\phantom{*} & -0.26 (0.13)* \\
AinS (popularity spread) & -0.87 (0.73)\phantom{*} & 1.82 (0.65)* &  & 0.02 (0.42)\phantom{*} & 0.67 (0.38)\phantom{*} &  & 0.57 (0.34)\phantom{*} & 0.45 (0.29)\phantom{*} \\
AoutS (activity spread) & 0.21 (0.57)\phantom{*} & -5.96 (2.79)* &  & 0.71 (0.40)\phantom{*} & -0.74 (0.80)\phantom{*} &  & 0.89 (0.37)* & 0.36 (0.31)\phantom{*} \\
AT-T (path closure) & 1.93 (0.73)* & 0.50 (0.37)\phantom{*} &  & 0.73 (0.23)* & 0.14 (0.20)\phantom{*} &  & 0.49 (0.26)\phantom{*} & 0.49 (0.25)\phantom{*} \\
A2P-T (multiple connectivity) & 2.58 (1.61)\phantom{*} & 0.37 (0.28)\phantom{*} &  & -0.22 (0.06)* & -0.06 (0.1)\phantom{*} &  & 0.35 (0.13)* & 0.36 (0.14)* \\ \\
\textbf{Actor-relation effects (exogenous)} & \multicolumn{1}{l}{} & \multicolumn{1}{l}{} &  & \multicolumn{1}{l}{} & \multicolumn{1}{l}{} &  & \multicolumn{1}{l}{} & \multicolumn{1}{l}{} \\
Age (difference) & 0.03 (0.04)\phantom{*} & -0.01 (0.02)\phantom{*} &  & 0.00 (0.01)\phantom{*} & 0.00 (0.01)\phantom{*} &  & -0.01 (0.02)\phantom{*} & -0.04 (0.01)* \\
Education level (difference) & -0.34 (0.23)\phantom{*} & 0.18 (0.13)\phantom{*} &  & 0.08 (0.07)\phantom{*} & -0.07 (0.08)\phantom{*} &  & -0.12 (0.11)\phantom{*} & -0.12 (0.11)\phantom{*} \\
Work experience (sender) & 0.00 (0.05)\phantom{*} & -0.12 (0.08)\phantom{*} &  & 0.02 (0.01)* & 0.02 (0.02)\phantom{*} &  & -0.02 (0.02)\phantom{*} & -0.05 (0.02)* \\
Job tenure (sender) & 0.06 (0.06)\phantom{*} & 0.18 (0.09)* &  & 0.00 (0.01)\phantom{*} & -0.01 (0.02)\phantom{*} &  & 0.04 (0.02)\phantom{*} & 0.07 (0.03)* \\
Openness (sender) & 1.19 (2.29)\phantom{*} & -3.92 (3.85)\phantom{*} &  & -0.35 (0.47)\phantom{*} & 0.13 (0.92)\phantom{*} &  & 0.06 (0.63)\phantom{*} & -0.46 (0.68)\phantom{*} \\
Openness (receiver) & 6.19 (2.58)* & -0.66 (1.29)\phantom{*} &  & -0.18 (0.67)\phantom{*} & 0.2 (0.69)\phantom{*} &  & -1.08 (0.64)\phantom{*} & 0.61 (0.59)\phantom{*} \\
Controlled motivation (sender) & -0.89 (1.88)\phantom{*} & -12.22 (4.84)* &  & 0.56 (0.68)\phantom{*} & 2.16 (1.34)\phantom{*} &  & 2.84 (1.01)* & 1.25 (0.85)\phantom{*} \\
Controlled motivation (receiver) & -2.44 (2.90)\phantom{*} & 1.61 (1.42)\phantom{*} &  & -1.53 (0.84)\phantom{*} & -1.23 (0.93)\phantom{*} &  & -2.69 (0.94)* & 0.67 (0.71)\phantom{*} \\
Autonomous motivation (sender) & 1.80 (2.12)\phantom{*} & -0.67 (2.32)\phantom{*} &  & 0.21 (0.52)\phantom{*} & -1.93 (1.11)\phantom{*} &  & -0.24 (0.88)\phantom{*} & -1.08 (0.7)\phantom{*} \\
Autonomous motivation (receiver) & 9.35 (3.03)* & -1.22 (1.12)\phantom{*} &  & 1.53 (0.75)* & 1.61 (0.90)\phantom{*} &  & 3.62 (1.12)* & -0.15 (0.69)\phantom{*} \\
Identification with group (sender) & 0.13 (1.13)\phantom{*} & -0.36 (1.54)\phantom{*} &  & 0.00 (0.33)\phantom{*} & 1.77 (0.79)* &  & -1.19 (0.48)* & -0.1 (0.42)\phantom{*} \\
Identification with group (receiver) & 2.05 (1.62)\phantom{*} & -0.91 (0.77)\phantom{*} &  & 0.02 (0.47)\phantom{*} & -0.08 (0.38)\phantom{*} &  & -0.08 (0.42)\phantom{*} & -0.11 (0.39)\phantom{*} \\
Employer (match) & 0.07 (0.86)\phantom{*} & 2.36 (0.89)* &  & 0.77 (0.32)* & 0.64 (1.05)\phantom{*} &  & 0.61 (0.35)\phantom{*} & 0.46 (0.31)\phantom{*} \\
Employer (mismatch reciprocity) & -8.74 (22.83)\phantom{*} & 3.79 (1.32)* &  & 0.71 (0.68)\phantom{*} & 0.48 (0.26)\phantom{*} &  & 1.92 (0.83)* & -0.42 (1.21)\phantom{*} \\ \\
\textbf{Dyadic covariate effects (exogenous)} & \multicolumn{1}{l}{} & \multicolumn{1}{l}{} &  & \multicolumn{1}{l}{} & \multicolumn{1}{l}{} &  & \multicolumn{1}{l}{} & \multicolumn{1}{l}{} \\
Cognition-based tust & 2.75 (0.79)* & 2.53 (0.56)* &  & 0.78 (0.21)* & 0.96 (0.40)* &  & 1.47 (0.33)* & 0.34 (0.28)\phantom{*} \\
Reporting hierarchy & 0.47 (0.81)\phantom{*} & 0.31 (0.68)\phantom{*} &  & -1.02 (0.48)* & -0.01 (0.05)\phantom{*} &  & 0.62 (0.50)\phantom{*} & 0.84 (0.4)* \\
Geographic proximity & -0.26 (0.13)* & 0.08 (0.09)\phantom{*} &  & 0.00 (0.04)\phantom{*} & 0.00 (0.00)\phantom{*} &  & -0.08 (0.05)\phantom{*} & -0.22 (0.05)* \\ \bottomrule
\end{tabular}

\begin{tablenotes}
\footnotesize
\item[a] Estimates are significant (*) when the absolute value of the parameter estimate is more than twice the magnitude of the estimated standard error.
\item[b] Goodness of fit scores for non-explicitly modelled statistics were less than 2 in all models, a, and less than 0.1 for all explicitly modelled effects, indicating the models provide adequate fit to most aspects of the social structure.

\end{tablenotes}

\end{threeparttable}
%
}
\end{sidewaystable}

\section{Models exploring motivation, trust and power}

The first set of models examined how autonomous motivation, cognition-based trust, and reporting hierarchy shape tacit and explicit knowledge sharing ties in each case. \medskip

Parameters used in the first set of models included controlled and autonomous sender/receiver effects (to determine to what extent the two types of motivation predict knowledge sharing) and cognition-based trust and reporting hierarchy dyadic covariate effects (to check to what extent knowledge sharing ties align with cognition-based trust ties and reporting hierarchy ties). Other parameters controlled for network structure (reciprocity, simple and multiple connectivity, popularity and activity spread, and transitive closure), homophily (age and education level difference, employer match), personality (openness to experience sender/receiver), social identity (identification with group sender/receiver), reciprocal exchanges across organisational boundaries (employer mismatch reciprocity), and geographic proximity (log of kilometre distance). Results for each case are presented side-by-side in Table \ref{tab:ergm_1}. \medskip

Apart from modelling each case separately, each case's networks were combined and modelled as one big network. Combining the tacit and explicit knowledge provider networks allows one to see which model parameters are significant across all three cases. Table \ref{tab:ergm_2} presents the results of the combined network analysis.

\subsection{Case 1: Cold-chain innovation}

Looking at the results for Case 1, parameter estimates for the tacit knowledge provider network show significant and positive effects for path closure (AT-T = 1.93), openness to experience (receiver = 6.19), autonomous motivation (receiver = 9.35), and cognition-based trust (dyadic covariate = 2.75). These effects suggest participants prefer to share their tacit knowledge with others they have strong ties with. The significant and positive receiver effects reflect a strong learning orientation, i.e. participants who are open to experience and autonomously motivated are more likely to seek out tacit knowledge. The significant and negative effect observed for geographic proximity (dyadic covariate = -0.26) indicates that tacit knowledge is more likely to be exchanged with others who are nearby.  \medskip

As to the explicit knowledge provider network, parameter estimates show significant and positive effects for popularity spread (AinS = 1.82), job tenure (sender = 0.18), employer (match = 2.36 and mismatch reciprocity = 3.79), and cognition-based trust (dyadic covariate = 2.53). In other words, participants are likely to direct explicit knowledge to a few actors they trust. Explicit knowledge tends to be provided by participants who have been in the job for some time. There is also a tendency to exchange explicit knowledge with others from the same organisation. That there is also a significant two-way exchange of explicit knowledge across organisational boundaries is a sign of good collaboration. The significant and negative effect for activity spread suggests explicit knowledge is being provided evenly by many participants, not by a few active individuals. Moreover, participants are not feeling pressured to provide explicit knowledge, judging from the significant and negative controlled motivation (sender) effect. Formal structure is not significant in the explicit knowledge provider network. This suggests that structure is not interfering with or inhibiting agency that much ( \medskip 

\subsection{Case 2: Farm system innovation}

Modelling of the tacit knowledge provider network yields significant and positive effects for path closure (AT-T = 0.73), work experience (sender = 0.02), autonomous motivation (receiver = 1.53), and cognition-based trust (dyadic covariate = 0.78). The effects for path closure and cognition-based trust indicate participants prefer to share their tacit knowledge with others they trust. Recipients of tacit knowledge also tend to be autonomously motivated. This suggests they have a strong learning orientation. There are significant and negative effects for both multiple connectivity (A2P = -0.22) and reporting hierarchy (dyadic covariate = -1.02). The combination of a significant and positive effect for path closure and a significant and negative effect for multiple connectivity is a sign of good collaboration. Participants are sufficiently well-connected that brokerage is no longer required. The significant and negative effect for reporting hierarchy indicates that tacit knowledge sharing happens mostly through informal structures. Geographic proximity is not a significant factor in the tacit knowledge provider network. \medskip

With respect to the explicit knowledge provider network, the estimates only show significant and positive effects for identification with group (sender = 1.77) and cognition-based trust (dyadic covariate = 0.96). Explicit knowledge is more likely to be provided to trusted others and those they identify strongly with. 

\subsection{Case 3: Global partnership for honeybee research}

Estimates for the tacit knowledge provider network show significant and positive effects for activity spread (AoutS = 0.89), multiple connectivity (A2P = 0.35), controlled motivation (sender = 2.81), autonomous motivation (receiver = 3.62), employer (mismatch reciprocity = 1.92), and cognition-based trust (dyadic covariate = 1.47). The significant and positive effect for activity spread suggests tacit knowledge is being provided by a relatively small number of participants. Participants are also more likely to share their tacit knowledge with others they trust. The significant and positive effect for multiple connectivity is an indicator of substantial brokerage. As with the other cases, recipients of tacit knowledge tend to be autonomously motivated. The significant and positive effect for controlled motivation (sender) suggests, contrary to expectations, that many participants feel obliged to share their tacit knowledge. There is a significant two-way exchange of tacit knowledge across organisational boundaries, judging by the significant and positive effect for employer (mismatch reciprocity). \medskip

In terms of the explicit knowledge provider network, the estimates show significant and positive effects for multiple connectivity (A2P = 0.36), job tenure (sender = 0.07), and reporting hierarchy (dyadic covariate = 0.84). In other words, much of the explicit knowledge flowing through the network is coming from participants who have been in their job for some time. Explicit knowledge also tends to flow up the reporting hierarchy. The estimates show significant and negative effects for simple connectivity (TwoPath = -0.26), age (difference = -0.04), work experience (sender = -0.05), and geographic proximity (dyadic covariate = -0.22). These indicate that received explicit knowledge is not passed on, age homophily is not a factor in explicit knowledge exchanges, and that more experienced participants are less likely to share explicit knowledge with others. Explicit knowledge is also more likely to be exchanged with nearby participants. 

\subsection{Analysis of the combined networks}

\begin{table}[]
\centering
\resizebox{0.9\textwidth}{!}{	
\begin{threeparttable}
\footnotesize
\setlength{\tabcolsep}{6pt}
\renewcommand{\arraystretch}{1}
\caption{ERGM parameter estimates for the combined tacit and explicit knowledge provider networks.}
\label{tab:ergm_2}
\begin{tabular}{@{}lrlr@{}}
\toprule
 & \multicolumn{3}{c}{Cases 1 + 2 + 3} \\ \cmidrule(l){2-4} 
 & \multicolumn{1}{c}{Tacit} &  & \multicolumn{1}{c}{Explicit} \\ \cmidrule(lr){2-2} \cmidrule(l){4-4} 
\textbf{Purely structural effects (endogenous)} &  &  &  \\
Arc (edge) & -4.66 (0.71)* &  & -2.95 (0.83)* \\
Reciprocity (mutuality) & -0.37 (0.39)\phantom{*} &  & -0.51 (0.37)\phantom{*} \\
TwoPath (simple connectivity) & -0.05 (0.04)\phantom{*} &  & -0.09 (0.06)\phantom{*} \\
AinS (popularity spread) & 0.01 (0.18)\phantom{*} &  & 0.63 (0.19)* \\
AoutS (activity spread) & 0.63 (0.18)* &  & 0.05 (0.21)\phantom{*} \\
AT-T (path closure) & 0.84 (0.17)* &  & 0.56 (0.13)* \\
A2P-T (multiple connectivity) & 0.01 (0.05)\phantom{*} &  & 0.09 (0.08)\phantom{*} \\
&  &  &  \\
\textbf{Actor-relation effects (exogenous)} &  &  &  \\
Age (difference) & -0.01 (0.01)\phantom{*} &  & -0.01 (0.01)\phantom{*} \\
Education level (difference) & 0.08 (0.04)* &  & 0.03 (0.04)\phantom{*} \\
Work experience (sender) & 0.00 (0.01)\phantom{*} &  & -0.01 (0.01)\phantom{*} \\
Job tenure (sender) & 0.00 (0.01)\phantom{*} &  & 0.01 (0.01)\phantom{*} \\
Openness (sender) & -0.53 (0.34)\phantom{*} &  & -0.33 (0.41)\phantom{*} \\
Openness (receiver) & -0.31 (0.39)\phantom{*} &  & 0.20 (0.32)\phantom{*} \\
Controlled motivation (sender) & 0.55 (0.57)\phantom{*} &  & 0.56 (0.62)\phantom{*} \\
Controlled motivation (receiver) & -0.76 (0.64)\phantom{*} &  & -0.32 (0.48)\phantom{*} \\
Autonomous motivation (sender) & -0.62 (0.39)\phantom{*} &  & -1.29 (0.44)* \\
Autonomous motivation (receiver) & 1.71 (0.54)* &  & 0.12 (0.39)\phantom{*} \\
Identification with group (sender) & 0.08 (0.23)\phantom{*} &  & 0.14 (0.27)\phantom{*} \\
Identification with group (receiver) & -0.32 (0.25)\phantom{*} &  & 0.25 (0.22)\phantom{*} \\
Employer (match) & 0.47 (0.16)* &  & 0.49 (0.16)* \\
Employer (mismatch reciprocity) & 0.77 (0.46)\phantom{*} &  & 1.07 (0.45)* \\
 &  &  &  \\
\textbf{Dyadic covariate effects (exogenous)} &  &  &  \\
Cognition-based trust & 1.26 (0.16)* &  & 0.89 (0.16)* \\
Reporting hierarchy & 0.30 (0.24)\phantom{*} &  & 0.82 (0.22)* \\
Geographic proximity & 0.00 (0.02)\phantom{*} &  & -0.07 (0.02)* \\ \bottomrule
\end{tabular}
\begin{tablenotes}
\footnotesize
\item[a] Estimates are significant (*) when the absolute value of the parameter estimate is more than twice the magnitude of the estimated standard error.
\item[b] Goodness of fit scores were less than 2 for more than 97.85\% of the non-explicitly modelled statistics, indicating the models provide adequate fit to most aspects of the knowledge exchange structure.
%\item[c] Given each partnership had a different number of participants (18, 25, and 40 participants in Cases 1, 2, and 3 respectively) and each partnership's maturity when surveyed, one must not read too much into these estimates.
\end{tablenotes}
\end{threeparttable}
}
\end{table}

Parameter estimates for the combined tacit knowledge network show that across all three cases, there is a significant and positive effect for activity spread and path closure (0.63 and 0.84, respectively). One can infer from this that knowledgeable people are happy to share their know-how with others in their group. The significant and positive effect for age difference (0.08) indicates that people are more likely to share knowledge with others of a similar age (across all three cases, there is age homophily effect). As with each case, the significant and positive autonomous motivation (receiver) effect (1.71) suggests most tacit knowledge recipients are autonomously motivated. The significant and positive employer (match) effect (0.47) indicates that people are inclined to share their tacit knowledge with others employed in the same organisation. People also tend to share their know-how with others they deem trustworthy (significant and positive cognition-based trust dyadic covariate effect = 1.26). \medskip

In contrast, parameter estimates for the combined explicit knowledge provider network show a significant and positive effect for popularity spread (0.63), indicating that across the three cases, explicit knowledge tends to flow to more central actors. Moreover, the significant and positive dyadic covariate effect for reporting hierarchy (0.82) suggests these central actors occupy senior positions. The significant and positive path closure effect (0.56) indicates that explicit knowledge sharing mostly occurs within well-connected groups. Interestingly, the significant and negative autonomous motivation (sender) effect (-1.29) suggests that across the three cases, autonomously motivated people are less likely to share explicit knowledge. Explicit knowledge tends to be shared with others employed by the same organisation (significant and positive employer (match) effect = 0.49). However, the significant and positive employer (mismatch reciprocity) effect (1.07) indicates that explicit knowledge exchanges among open innovation partners tends to be reciprocal. People are also inclined to share explicit knowledge with people they trust (significant and positive dyadic cognition-based trust covariate effect = 0.89) and who are nearby (significant and negative geographic proximity dyadic covariate effect = -0.07).  

\section{Models examining broker roles}

The second set of models examining the significance of broker roles in each case used fewer parameters than the first set, just path closure and the five broker roles. Path closure was included to highlight the tension between network closure and brokerage. Results from the second set of modelling are presented in Table \ref{tab:ergm_3}. These must be considered in light of the broker role counts presented in Figure \ref{fig:gf_brokerage} and the results from the first set of models. Despite including only the five broker roles plus an edge and/or path closure effect, the models assessing the significance of broker roles are able explain most of the observed network configurations. It appears one can characterise open innovation partnerships in terms of path closure and broker roles alone. \medskip 

\begin{sidewaystable}[p]
\centering
\resizebox{0.9\textwidth}{!}{%	
\begin{threeparttable}
\footnotesize
\setlength{\tabcolsep}{6pt}
\renewcommand{\arraystretch}{1}
\caption[Parameter estimates for the second set of ERGMs]{ERGM parameter estimates for models assessing broker roles. Refer to Table \ref{tab:ergm_params} for an explanation of network parameters.}
\label{tab:ergm_3}
\begin{tabular}{@{}lrrlrrlrr@{}}
\toprule
\multicolumn{1}{c}{} & \multicolumn{2}{c}{Case 1} & \multicolumn{1}{c}{} & \multicolumn{2}{c}{Case 2} & \multicolumn{1}{c}{} & \multicolumn{2}{c}{Case 3} \\ \cmidrule(lr){2-3} \cmidrule(lr){5-6} \cmidrule(l){8-9} 
\multicolumn{1}{c}{} & \multicolumn{1}{c}{Tacit} & \multicolumn{1}{c}{Explicit} & \multicolumn{1}{c}{} & \multicolumn{1}{c}{Tacit} & \multicolumn{1}{c}{Explicit} & \multicolumn{1}{c}{} & \multicolumn{1}{c}{Tacit} & \multicolumn{1}{c}{Explicit} \\ \cmidrule(lr){2-3} \cmidrule(lr){5-6} \cmidrule(l){8-9} 
\textbf{Purely structural effects (endogenous)} & \multicolumn{1}{l}{} & \multicolumn{1}{l}{} &  & \multicolumn{1}{l}{} & \multicolumn{1}{l}{} &  & \multicolumn{1}{l}{} & \multicolumn{1}{l}{} \\
Arc (edge) & -2.56 (0.23)* & -2.47 (0.29)* &  & -2.52 (0.31)* & -2.34 (0.31)* &  & -3.13 (0.12)* & -3.09 (0.15)* \\
ATA-T (path closure) & 0.73 (0.34)* & 0.85 (0.14)* &  & 1.06 (0.12)* & 0.57 (0.12)* &  & \makecell[c]{---} & 0.81 (0.11)* \\ \\
\textbf{Actor-brokerage effects (exogenous)} & \multicolumn{1}{l}{} & \multicolumn{1}{l}{} &  & \multicolumn{1}{l}{} & \multicolumn{1}{l}{} &  & \multicolumn{1}{l}{} & \multicolumn{1}{l}{} \\
$w_O$ (liaison) & -0.39 (0.19)\phantom{*} & -0.11 (0.08)\phantom{*} &  & -0.22 (0.06)* & -0.04 (0.06)\phantom{*} &  & 0.04 (0.06)\phantom{*} & -0.2 (0.09)* \\
$w_I$ (coordinator) & -0.15 (0.46)\phantom{*} & -0.1 (0.29)\phantom{*} &  & -0.05 (0.19)\phantom{*} & -0.09 (0.19)\phantom{*} &  & 0.04 (0.06)\phantom{*} & -0.01 (0.04)\phantom{*} \\
$b_{OI}$ (gatekeeper) & -0.55 (0.99)\phantom{*} & 0.07 (0.12)\phantom{*} &  & -0.17 (0.11)* & -0.02 (0.11)\phantom{*} &  & 0.32 (0.05)* & 0.06 (0.05)\phantom{*} \\
$b_{IO}$ (representative) & 0.42 (0.56)* & -0.07 (0.14)\phantom{*} &  & -0.19 (0.1)* & 0.01 (0.1)\phantom{*} &  & 0.33 (0.03)* & 0.05 (0.06)\phantom{*} \\
$b_O$ (itinerant broker) & -0.63 (0.46)\phantom{*} & -0.62 (0.34)\phantom{*} &  & -0.15 (0.11)\phantom{*} & -0.07 (0.11)\phantom{*} &  & -1.54 (0.48)* & -0.42 (0.23)\phantom{*} \\
 \bottomrule
\end{tabular}
\begin{tablenotes}
\footnotesize
\item[a] Estimates are significant (*) when the absolute value of the parameter estimate is more than twice the magnitude of the estimated standard error.
\item[b] Path closure was not modelled in the Case 3 tacit knowledge network. Inclusion of this parameter resulted in a degenerate model. Despite this, the goodness-of-fit in this particular model was excellent (t-ratios $<2$ for 97\% of the non-explicitly modelled statistics).
\item[c] Goodness of fit scores were less than 2 for 98.3\% of the non-explicitly modelled statistics, and less than 0.1 for all explicitly modelled parameters, indicating the models provide adequate fit to most aspects of the social structure.

\end{tablenotes}

\end{threeparttable}
%
}
\end{sidewaystable}

\subsection{Case 1: Cold-chain innovation}

The parameter estimates presented in Table \ref{tab:ergm_1} indicate brokerage does not feature strongly in either the tacit and explicit knowledge networks.  Looking at the broker role counts presented in Figure \ref{fig:gf_brokerage}, the representative role dominates the broker role counts for the tacit knowledge network. The parameter estimates presented in Table \ref{tab:ergm_2} confirm that the representative role is significant in the tacit knowledge network ($b_{IO}$ = 0.42). It appears participants are happy to pass on internal tacit knowledge to others working for different organisations. Though the liaison role dominates the broker role counts in the explicit knowledge network, the parameter estimates indicate this role is not significant. \medskip

Whereas the models exploring autonomous motivation, cognition-based trust, and reporting hierarchy show a significant and positive effect for path closure in the tacit knowledge network only, the more simple models examining broker roles show a significant and positive effect for path closure in both the tacit and explicit knowledge networks (AT-T = 0.73 and 0.85, respectively). Path closure was included to highlight the tension between network closure and brokerage. The significant and positive effect for path closure indicates network closure is more dominant than brokerage in both networks. 

\subsection{Case 2: Farm system innovation}

The parameter estimates presented in Table \ref{tab:ergm_1} show that there is significantly less brokerage and significantly more clustering in the tacit knowledge network. This combination is a sign of excellent collaboration. Given this case was winding up at the time of data collection, this is not surprising. Brokers are unlikely to feature in well-established partnerships. Although the liaison role domineers broker role counts in both the tacit and explicit knowledge networks, the modelling of broker roles shows this role is under-represented in the tacit knowledge network ($w_0$ = -0.22) and not significant in the explicit knowledge network. The modelling also reveals that the gatekeeper and representative roles are significantly under-represented in the tacit knowledge network ($b_{IO}$ = -0.17, $b_{OI}$ = -0.19). 

\subsection{Case 3: Global partnership for honeybee research}

The parameter estimates presented in Table \ref{tab:ergm_1} suggest brokerage is significant in both the tacit and explicit knowledge networks. Considering this case was at a very early stage at the time of data collection, this is not unexpected. The absence of significant and positive effects for path closure in both networks suggests brokerage is the dominant network process. However, the parameter estimates presented in Table \ref{tab:ergm_2} are contradictory. These show a significant and positive effect for path closure in the explicit knowledge network (AT-T = 0.81). Unfortunately, path closure could not be modelled in the tacit knowledge network. Inclusion of this parameter resulted in a degenerate model. The significant and positive effect for path closure in the explicit knowledge network does indicate path closure is the more dominant network process. \medskip 

Looking at the broker role counts presented in Figure \ref{fig:gf_brokerage}, the itinerant broker role hardly features in either the tacit or explicit knowledge networks. Other roles are evenly spread in the tacit knowledge network, while the coordinator role dominates the explicit knowledge network, the modelling of broker roles show a significant and positive effect for both the gatekeeper and representative role in the tacit knowledge network ($b_{IO}$ = 0.32, $b_{OI}$ = 0.33). The modelling confirms that the itinerant broker role is significantly under-represented in the tacit knowledge network ($b_O$ = -1.54). Apart from the significant and negative effect for the liaison role ($w_O$ = -0.2), there are no significant broker role effects in the explicit knowledge network. \medskip  

\section{Summary}

Though each open innovation partnership is different, some network effects are common to two or more cases. Concerning the first set of models, the tacit knowledge provider networks exhibit significant path closure in Case 1 and Case 2. Path closure indicates that know-how features strongly in group practice. Tacit knowledge brokerage is significantly under-represented in Case 2 and significantly over-represented in Case 3 (. The opposite effects reflect the maturity of each partnership. Case 2 is quite mature with well-established groups, unlike Case 3, which is still in its formative stages. \medskip

All three cases have a significant autonomous motivation receiver effect in their tacit knowledge provider networks, indicating that learning features prominently in all three cases. Cognition-based trust is a significant dyadic covariate in all but one of the knowledge provider networks (it was not significant in the Case 3 explicit knowledge provider network). Participants are more likely to share their know-how and know-what with others they trust. \medskip

Concerning the second set of models assessing broker roles, path closure was significant in all but one of the knowledge provider networks (it was not possible to model path closure in the Case 3 tacit knowledge provider network). All three tacit knowledge provider networks had significant representative broker role effects. More specifically, Case 1 and 3 had a positive effect, unlike Case 2, which had a negative effect. Significant negative and positive gatekeeper role effects feature in Case 2 and Case 3 tacit knowledge provider networks, respectively. The representative and gatekeeper broker roles appear to be less critical in established tacit knowledge networks. \medskip

From a critical realist perspective, the exponential random graph modelling accounts for the observed or experienced reality. The next chapter presents the results from the qualitative analysis, which should shed light on contextual factors that influenced knowledge sharing (what happens in actual practice).